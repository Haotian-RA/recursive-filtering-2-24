\section{Introduction}
\label{sec:introduction}

The inherent dependency of recursive filter limits its computation speed.
To overcome the difficulty, the efficient way is to solve the recursive equation by seperating into
particular and homogeneous solutions.
One previous method based on SIMD operation includes block filtering \cite{Sung_86}, 
which basically computes consecutive samples in one SIMD vector by matrix multiplication. 
The number of arithmetric operation per sample is regardless of the block length. Thus, block filtering achieves
intra-block parallelism. 

One more efficient method exploits inter-block parallelism is to solve the particular part of recursive equation via multiple blocks \cite{Jaewoo_09}.
the consecutive samples are partitioned into multiple blocks in the distance of block length, thus, the multi-block filtering method
processes a squared matrix of samples in transposed grid. Compared with block filtering, the number of arithmetic operation per sample for multi-block filtering is decreased 
by the reciprocal of the block length.

However, there is a gap that most of the literatures ignore the rounding error issue happens at higher order recursive filter when implementing the algorithms in direct form.
Since the poles depend on all coefficients of recursive equation, the finite precision of rounded coefficients may cause some poles near the unit circle
moving outside the unit circle. The solution is to breaking up the higher order filter into cascaded sections, thus the poles are only affected by few coefficients to mitigate
the rounding error issue.

To address both accuracy and efficiency, this paper implements the recursive filter by cascading second order sections with a novel multi-block filtering method applied.
Specifically, compared with \cite{Jaewoo_09}, we extend the multi-block filtering method on solving both particular and homogeneous parts. With the idea of recursive doubling \cite{Kogge_73},
the proposed algorithm solves for the homogeneous part very efficiently. More importantly, the algorithm behind this paper is very suitable for cascaded form of recursive filter.


