\section{Implementation on Software-defined Radio}%
\label{sec:implementation_on_SDR}

\begin{figure}[t]
    \centerline{\includegraphics[width=3.4in]{SDR_receiver_horizontal.png}}
    \caption{Block diagram for implementing the proposed algorithm in TBB}
    \label{fig:SDR_receiver}
    \end{figure}

% To make our proposed algorithm work at a very high sample rate,
% the first improvement of algorithm is by using pipeline.
% Threading Building Blocks (TBB) is a well-known C++
% library that enables parallel programming on multicore processor~\cite{Michael_19}.
To demonstrate the practicality of our algorithms, we have implemented them in C++ on a general
purpose processor (GPP).
The different aspects of the algorithm are mapped to logical nodes in a pipelined, parallel processing architecture
using Threading Building Blocks (TBB)~\cite{Michael_19}.

Fig.~\ref{fig:SDR_receiver} shows a simple block diagram
illustrating the pipeline for the complete joint detection and
estimation process.
Due to space constraints, we cannot give details for each node; the
computations performed by the nodes relate closely to the expressions
throughout this paper
% some of which are
as
indicated in Fig.~\ref{fig:SDR_receiver}.

It is necessary to discuss the node for computing the phasor estimate $\hat{\xi}$ (node V). 
The numerator of~\eqref{eq:opt_xi} performs a time-varying
convolution,
which cannot be computed efficiently via FFT.
Our solution to increasing the computational efficiency is to further
parallelize this block.
Specifically,~\eqref{eq:opt_xi} can be computed in three stages:
1.~Compute  segments of $\sum r_ns_n^*$ in  parallel;
2.~still in parallel, each segment is then phase-shifted by $\hat{\delta}_{SL}$
multiplied by the middle index of the 
segment.
3. Sum all segments together.
In other words, \eqref{eq:opt_xi} is computed more efficiently as
\begin{equation}
  \label{eq:refined_opt_S}
  ||\bm{s}||^2\cdot\hat{\xi} \approx \sum_{i=0}^{L-1} e^{-j\pi \hat{\delta}\frac{N(2i+1)}{L}}
  \sum_{n=iN/L}^{(i+1)N/L-1}r_ns_n^*,
\end{equation}
where $L$ is the number of segments processed in parallel.
As a result,
the efficiency of computing $\hat{\xi}$ is improved by $L^2$.

% We now show the results of implementing the proposed algorithm
% by using
Our goal is to assess the highest sample rate that a
highly parallel, pipelined implementation of our algorithms can sustain. 
The signals are transmitted and received between two universal software radio peripherals (USRP);
these are connected by \SI[per-mode=symbol]{5}{\giga\bits\per\second} Ethernet cables to
laptops.
On the receiver side, 
the CPU includes 6 cores and 12 threads with a \SI{4.5}{\giga\hertz}
clock frequency.

Each node in Fig.~\ref{fig:SDR_receiver} was benchmarked (using Google benchmark) to measure their throughput.
The results for each node are shown 
in Table~\ref{table:BM_function_nodes}.
The times displayed in the table
refer to processing a  buffer of size 512 samples.
The throughput of the 
pipeline is dominated by the node with the longest execution time.
Thus, the  throughput of our current implementation is bounded by
$\SI{512}{\sample}/\SI{52343}{\nano\second}  \approx \SI[per-mode=symbol]{9.78}{\mega\sample\per\second}$.
In operation, the throughput of the algorithm was measured to be in
the range of
\SI[per-mode=symbol]{9.5}{\mega\sample\per\second} to \SI[per-mode=symbol]{10.5}{\mega\sample\per\second} with latency near \SI{292}{\micro\second}.
The detection algorithm is very robust and the false alarm probability is near 0 at moderate SNR.

\begin{table}[t]
    \caption{Benchmark results of nodes in Fig.~\ref{fig:SDR_receiver} with buffer size 512}  % title of Table
    \centering % used for centering table
    \begin{tabular}{c c c c} % centered columns (4 columns)
    \hline\hline %inserts double horizontal lines
    Node name & Time (ns) & CPU (ns) & Iterations \\ [0.5ex] % inserts table
    %heading
    \hline % inserts single horizontal line
    I. Buffer  & 3885 & 3885 & 200135 \\ % inserting body of the table
    II. $r_m^*r_{m-k}$  & 3252 & 3252 & 216211 \\
    III. $r_m^*r_{m-k}s_{m-k}^*s_m$ & 52343 & 52342 & 13388 \\
    IV. $\hat{\delta}_{SL}^{(1)}$ & 8432 & 8432 & 86480 \\
    V. $\hat{\xi}$ & 52309 & 52309 & 13359 \\
    VI. $||\bm{r}||$ & 19638 & 19638 & 34872 \\ % [1ex] adds vertical space
    VII. $\rho(p)$ & 16217 & 16217 & 42283 \\
    VIII. detection & 3669 & 3669 & 199976 \\
    IX. carrier synchronization & 3075 & 3075 & 222154  \\ [1ex]
    \hline
    \end{tabular}
    \label{table:BM_function_nodes} % is used to refer this table in the text
  \end{table}

