\documentclass[10pt,final,conference,comsoc]{IEEEtran}

\usepackage[utf8]{inputenc}
\usepackage[T1]{fontenc} % optional
\usepackage{amsmath,amssymb,amsfonts}
\usepackage[cmintegrals]{newtxmath}
\usepackage{cite}
\usepackage{algorithm}
\usepackage{graphicx}
\usepackage{textcomp}
\usepackage{xcolor}
\usepackage{bm}
\usepackage{siunitx}
\usepackage{comment}
\usepackage{mathtools}
\usepackage{url}
\usepackage{arydshln}
\usepackage{algpseudocode}


\makeatletter
\newcommand*\dashline{\rotatebox[origin=c]{90}{$\dabar@\dabar@\dabar@$}}
\makeatother

% \newcommand{\LRT}[2]{\mathrel{\mathop\gtrless\limits^{#1}_{#2}}}
% \newcommand{\cn}{\mathcal{CN}}
% \newcommand{\n}{\mathcal{N}}
% \newcommand{\D}{\mathcal{D}}

\def\BibTeX{{\rm B\kern-.05em{\sc i\kern-.025em b}\kern-.08em
    T\kern-.1667em\lower.7ex\hbox{E}\kern-.125emX}}

\graphicspath{{../figures/}{../matlab/}}

\DeclareUnicodeCharacter{0303}{}

% \DeclareSIUnit{\sample}{S}
% \DeclareSIUnit{\bits}{b}

\begin{document}

%\title{Low Complexity Methods for Joint Detection and Synchronization of TDMA Bursts}
\title{\huge An Accurate and Efficient Implementation of Recursive Filtering based on SIMD and Cascaded Form }
\author{\IEEEauthorblockN{Haotian Zhai and Bernd-Peter Paris}
\IEEEauthorblockA{\textit{Department of Electrical and Computer Engineering} \\
\textit{George Mason University}\\
Fairfax, VA 22030 \\
\{hzhai,pparis\}@gmu.edu}}

\maketitle

\begin{abstract}
  % To operate at low SNR, LPI/LPD communication systems rely on
  % coherent processing which requires that the receiver is precisely
  % synchronized in time, frequency and phase. 
  % To that end, this paper proposes a family of data-aided joint
  % frame and carrier synchronization algorithms.
  % A sequential detection algorithm based on the generalized likelihood ratio test (GLRT)
  % is used to detect an embedded preamble signal in the received data
  % stream.
  % Novel algorithms for low-complexity, coarse carrier synchronization
  % at low SNR provide carrier estimates
  % during sequential detection.
  % After detection,
  % the coarse carrier estimate is refined for use in coherent demodulation.
  % The proposed family of algorithms can be scaled to support operation
  % over a wide range of SNR, including SNR below \SI{0}{\decibel}.
  % Algorithms are validated through simulation.
  % The practicality of our approach is demonstrated by real-time operation on
  % a standard SDR platform with sample rates approaching
  % \SI{10}{\mega\hertz}.
\end{abstract}

\begin{IEEEkeywords}
% LPI/LPD, carrier synchronization, frame synchronization, sequential
% detection, GLRT, SDR
\end{IEEEkeywords}

\section{Introduction}
\label{sec:introduction}

Recursive filters are used frequently in digital signal processing.
The main advantage of a recursive filter is the low computation efficiency in implementation.
In order to meet the same specification in terms of passband, stopband, ripple, or roll-off,
recursive filter needs a smaller number of taps than non-recursive filter, which implies
a correspondingly fewer number of computations per unit time. 
The computational savings is often of a rather large factor.

However,
the inherent dependency of recursive filter limits its computation speed.
To overcome the difficulty, the normal way is to dealing with the recursive equation by seperating into
particular and homogeneous solutions.
One previous method based on SIMD operation includes block filtering \cite{Sung_86}, 
which basically computes consecutive samples in one SIMD vector by matrix multiplication. 
The number of arithmetic operation per sample is regardless of the block length. Thus, the block filtering method achieves
intra-block parallelism. 

One more efficient algorithm that exploits inter-block parallelism is to solve the particular part of recursive equation via multiple blocks \cite{Jaewoo_09}.
The consecutive samples are partitioned into multiple blocks in the distance of block length.
% , thus, the multi-block filtering method processes a squared matrix of samples in transposed grid. 
Compared with block filtering, the number of arithmetic operation per sample for multi-block filtering is decreased 
by the reciprocal of the block length. 

There are also many algorithms proposed for efficiently implementing the recursive filter, e.g., including
space-time transformation of loop iterations \cite{Schaffer_03}, tensor product representation \cite{Robelly_04} and
poles reconfiguration \cite{Liu_16}.
However, there is a gap that most of the literatures ignore the rounding error issue \cite{Christoph_97} happens at higher order recursive filter when implementing the algorithms in direct form.
Since the poles depend on all the coefficients of recursive equation in direct form, the finite precision of rounded coefficients may cause some poles near the unit circle
moving outside the unit circle. The solution is to breaking up the higher order filter into cascaded sections, thus the poles are only affected by few coefficients to mitigate
the rounding error issue.

To address both accuracy (filter stability) and efficiency, 
this paper proposes an algorithm of forming the recursive filter by cascading second order sections
align with a novel multi-block filtering approach.
Specifically, compared with \cite{Jaewoo_09}, we extend the multi-block filtering method on solving both particular and homogeneous parts. With the idea of recursive doubling \cite{Kogge_73},
the proposed algorithm solves for the homogeneous part very efficiently. More importantly, the algorithm behind this paper is very suitable for cascaded form of recursive filter.




\section{The Proposed Algorithm}
\label{sec:algorithm}

The form of a general second order IIR filter is given by

\begin{equation}
    \label{eq:recursive_filter}
    % y[n] = x[n] + b_1x[n-1] + b_2x[n-2] + a_1y[n-1] + a_2y[n-2]
    y_n = x_n + b_1x_{n-1} + b_2x_{n-2} + a_1y_{n-1} + a_2y_{n-2}
\end{equation}
where $x_n$ and $y_n$ denote the input and output, respectively.
% missing some contents
The proposed algorithm in this paper processes multiple blocks (SIMD vectors) of data simultaneously
so that it utilizes both intra-block and inter-block parallelism. Define $M$ be the length of SIMD vector.
The input data of multiple blocks is exactly distributed in a squared matrix ($M$ by $M$), i.e.,

% a good example in this paper 
% \begin{equation}
% \left[\begin{array}{cccc|c}
% a_{11}&a_{12}&\cdots &a_{1n}&b_1\\
% a_{21}&a_{22}&\cdots &a_{2n}&b_2\\
% \vdots & &\ddots &\vdots \\
% a_{n1}&a_{n2}&\cdots &a_{nn}&b_n\\
% \end{array}\right]
% \end{equation}

% $$
% C=\left[
% \begin{array}{c ;{2pt/2pt} c}
% A & B \\ \hdashline[2pt/2pt]
% C & D
% \end{array}\right]
% $$

\begin{equation*}
    \label{eq:matrix_X}
    % \bm{X} = \left[\begin{array}{c|c|c|c}
    % x[0] & x[M] & \cdots & x[M^2{-}M] \\ 
    % x[1] & x[M{+}1] & \cdots & x[M^2{-}M{+}1] \\
    % \vdots & \vdots & \ddots & \vdots \\
    % x[M{-}1] & x[2M{-}1] &\cdots & x[M^2{-}1] \\
    % \end{array}\right]
    \bm{X} = \left[\begin{array}{c|c|c|c}
        x_0 & x_M & \cdots & x_{M^2-M} \\ 
        x_1 & x_{M+1} & \cdots & x_{M^2-M+1} \\
        \vdots & \vdots & \ddots & \vdots \\
        x_{M-1} & x_{2M-1} &\cdots & x_{M^2-1} \\
        \end{array}\right]
\end{equation*}
where each block of data is aligned with column and seperated by line.
We use $\bm{X}$ to denote the matrix version of input data. Furthermore, $\bm{X}[n]$
means the $n$th block and $\bm{X}[n][m]$ means the $m$th sample at $n$th block
of input matrix $\bm{X}$.
% missing some contents talking about dependency

To deal with the dependency problem of second order IIR filter equation, the algorithm
that computes the solution of finite impulse response (FIR), the particular and homogeneous solutions apart and 
add them to get the complete solutions is proposed.
% about homo and particular solutions and fir

\subsection{Finite Impulse Response (FIR)}

To compute \eqref{eq:recursive_filter}, we start with 

\begin{equation}
    \label{eq:fir_filter}
    v_n = x_n + b_1x_{n-1} + b_2x_{n-2} 
\end{equation}
where $v_n$ is the output of second order FIR filter. 

\subsubsection{Non-transposed multi-block filtering of FIR}

It is easy to apply the multi-block filtering method on FIR filter 
because of no dependency. The algorithm that computes the output of second order FIR filter 
with non-transposed matrix of data is given by

\begin{equation}
    \label{eq:FIR_block_filtering_wo_trans}
    \begin{aligned}
        % \bm{V} &{=} \left[\begin{array}{c|c|c}
        %     V[0] & \cdots & V[M^2{-}M] \\ 
        %     V[1] & \cdots & V[M^2{-}M{+}1] \\
        %     \vdots & \ddots & \vdots \\
        %     V[M{-}1] &\cdots & V[M^2{-}1] \\
        %     \end{array}\right]
        % = \bm{X} + b_1\bm{X}_{-1} + b_2\bm{X}_{-2} \\
        % &{=} \bm{X} + b_1\left[\begin{array}{c|c|c}
        %     V_0 & \cdots & V_{M^2{-}M} \\ 
        %     V_1 & \cdots & V_{M^2{-}M{+}1} \\
        %     \vdots & \ddots & \vdots \\
        %     V_{M{-}1} &\cdots & V_{M^2{-}1} \\
        %     \end{array}\right] +
        %     b_1\left[\begin{array}{c|c|c}
        %         V_0 & \cdots & V_{M^2{-}M} \\ 
        %         V_1 & \cdots & V_{M^2{-}M{+}1} \\
        %         \vdots & \ddots & \vdots \\
        %         V_{M{-}1} &\cdots & V_{M^2{-}1} \\
        %         \end{array}\right]
        \bm{V} &= \left[\begin{array}{c|c|c|c}
            v_0 & v_M & \cdots & v_{M^2-M} \\ 
            v_1 & v_{M+1} & \cdots & v_{M^2-M+1} \\
            \vdots & \vdots & \ddots & \vdots \\
            v_{M-1} & v_{2M-1} &\cdots & v_{M^2-1} \\
            \end{array}\right] \\
            &= \bm{X} + b_1\left[\begin{array}{c|c|c|c}
                x_{-1} & x_{M-1} & \cdots & x_{M^2-M-1} \\ 
                x_0 & x_M & \cdots & x_{M^2-M} \\
                \vdots & \vdots & \ddots & \vdots \\
                x_{M-2} & x_{2M-2} &\cdots & x_{M^2-2} \\
                \end{array}\right]  \\
            &+ b_2\left[\begin{array}{c|c|c|c}
                x_{-2} & x_{M-2} & \cdots & x_{M^2-M-2} \\ 
                x_{-1} & x_{M-1} & \cdots & x_{M^2-M-1} \\
                \vdots & \vdots & \ddots & \vdots \\
                x_{M-3} & x_{2M-3} &\cdots & x_{M^2-3} \\
                \end{array}\right]  \\
            % &= \bm{X} + b_1\bm{X}_{-1} + b_2\bm{X}_{-2}
    \end{aligned}
\end{equation}
% where $\bm{X}_{-k}[n] = \left[ x_{nM-k} ~ x_{nM-k+1} ~ \cdots ~ x_{(n+1)M-k-1}\right]^T$ denotes
% the $n$th block of $\bm{X}$ shuffled with the previous $k$ input data. 
where $\bm{V}$ denotes the matrix version of output data of FIR.
\eqref{eq:FIR_block_filtering_wo_trans}
can be computed among multiple blocks exploiting both intra-block and inter-block parallelism. The execution
procedure of SIMD operation is as follows:

1. Load and broadcast two coefficients $b_1$ and $b_2$ in SIMD vector.

2. Load the first block (SIMD vector) of input $\bm{X}$ and shuffle it twice by $x_{-1}$, $x_{-2}$ to get
the initial conditions. 

3. Compute two fused multiply and add (FMA) operations to get the first output block $\bm{V}[0]$.

4. Repeat step 2 and 3 by shuffling with the last two input data of previous block for $M{-}1$ times 
to get the entire $\bm{V}$.

5. Store the last block of $\bm{X}$ in the shift register as the initial conditions for next FIR filtering.

\subsubsection{Transposed multi-block filtering of FIR}
Next, We provide another solution to computing the output of second order FIR filter, which is based on transposed
matrix version of data. The algorithm also exhibits both intra-block and inter-block parallelism and is given by

\begin{equation}
    \label{eq:FIR_block_filtering_w_trans}
    \begin{aligned}
        % \bm{V} &{=} \left[\begin{array}{c|c|c}
        %     V[0] & \cdots & V[M^2{-}M] \\ 
        %     V[1] & \cdots & V[M^2{-}M{+}1] \\
        %     \vdots & \ddots & \vdots \\
        %     V[M{-}1] &\cdots & V[M^2{-}1] \\
        %     \end{array}\right]
        % = \bm{X} + b_1\bm{X}_{-1} + b_2\bm{X}_{-2} \\
        % &{=} \bm{X} + b_1\left[\begin{array}{c|c|c}
        %     V_0 & \cdots & V_{M^2{-}M} \\ 
        %     V_1 & \cdots & V_{M^2{-}M{+}1} \\
        %     \vdots & \ddots & \vdots \\
        %     V_{M{-}1} &\cdots & V_{M^2{-}1} \\
        %     \end{array}\right] +
        %     b_1\left[\begin{array}{c|c|c}
        %         V_0 & \cdots & V_{M^2{-}M} \\ 
        %         V_1 & \cdots & V_{M^2{-}M{+}1} \\
        %         \vdots & \ddots & \vdots \\
        %         V_{M{-}1} &\cdots & V_{M^2{-}1} \\
        %         \end{array}\right]
        \bm{V}^T &= \left[\begin{array}{c|c|c|c}
            v_0 & v_1 & \cdots & v_{M-1} \\ 
            v_M & v_{M+1} & \cdots & v_{2M-1} \\
            \vdots & \vdots & \ddots & \vdots \\
            v_{M^2-M} & v_{M^2-M+1} &\cdots & v_{M^2-1} \\
            \end{array}\right] \\
            &= \bm{X}^T + b_1\left[\begin{array}{c|c|c|c}
                x_{-1} & x_0 & \cdots & x_{M-2} \\ 
                x_{M-1} & x_M & \cdots & x_{2M-2} \\
                \vdots & \vdots & \ddots & \vdots \\
                x_{M^2-M-1} & x_{M^2-M} &\cdots & x_{M^2-2} \\
                \end{array}\right]  \\
            &+ b_2\left[\begin{array}{c|c|c|c}
                x_{-2} & x_{-1} & \cdots & x_{M-3} \\ 
                x_{M-2} & x_{M-1} & \cdots & x_{2M-3} \\
                \vdots & \vdots & \ddots & \vdots \\
                x_{M^2-M-2} & x_{M^2-M-1} &\cdots & x_{M^2-3} \\
                \end{array}\right]  \\
            % &= \bm{X}^T + b_1\bm{X}_{-1}^T + b_2\bm{X}_{-2}^T
    \end{aligned}
\end{equation}

Compared with non-transposed data matrix, the initial conditions for each block 
already exist at the previous two blocks except for the first block. Thus, to compute
\eqref{eq:FIR_block_filtering_w_trans}, only two shuffles are needed to provide the initial
conditions for the first block. The execution procedure is as follows:

1. Load and broadcast two coefficients $b_1$ and $b_2$ in SIMD vector.

2. Load the last two blocks of input $\bm{X}^T$ and shuffle those by $x_{-2}$, $x_{-1}$ respectively to get
the initial conditions for the first block.

3. Apply two FMA operations among the current block and the previous two blocks for $M$ times to get the entire $\bm{V}^T$.

4. Store (another shuffle) the last two input data located at the last two blocks in the shift register as the initial conditions for next FIR filtering.

\subsection{Infinite Impulse Response (IIR)}

After computing \eqref{eq:fir_filter}, the general equation of second order IIR filter in \eqref{eq:recursive_filter} is
simplified as

\begin{equation}
    \label{eq:iir_filter}
    y_n = v_n + a_1y_{n-1} + a_2y_{n-2} 
\end{equation}
We extend \eqref{eq:iir_filter} for the first couple of samples as follows:

\begin{equation}
    \label{eq:iir_filter_with_first_samples}
    \begin{aligned}
    & y_0 = \fbox{$v_0$} + a_1y_{-1} + a_2y_{-2} \\
    & y_1 = v_1 + a_1y_0 + a_2y_{-1} = \fbox{$v_1 + a_1v_0$} + (a_1^2+a_2)y_{-1}{+}a_1a_2y_{-2} \\
    & y_2 = v_2 + a_1y_1 + a_2y_0 = \fbox{$v_2 + a_1(v_1+a_1v_0) + a_2v_0$} + \\
    & \quad\quad\quad\quad\quad\quad\quad\quad\quad a_1^2(a_1y_{-1}+a_2y_{-2})+2a_1a_2y_{-1}+a_2^2y_{-2} \\
    & \quad\quad\quad\quad\quad\quad\quad\quad\quad\quad\quad\quad \vdots
    \end{aligned}
\end{equation}
where the terms in rectangle denote the particular solution and
the rest of terms are the homogeneous part of the second order IIR equation.
% missing some contents of literature
In this paper, we first calculate the particular solution by zeroing the homogenous part, where
the process is called zero initial condition (ZIC). After that, we correct the initial conditions (homogeneous part) for
getting the complete output. The latter process is called initial condition correction (ICC). Note, the above algorithms
are all processed by multi-block filtering method.

\subsubsection{Transposed zero initial condition (ZIC)}

From \eqref{eq:iir_filter_with_first_samples}, the algorithm that computes the particular solution 
of second order IIR filter with transposed matrix of data is given by

\begin{equation}
    \label{eq:ZIC_w_trans}
    \begin{aligned}
        % \bm{V} &{=} \left[\begin{array}{c|c|c}
        %     V[0] & \cdots & V[M^2{-}M] \\ 
        %     V[1] & \cdots & V[M^2{-}M{+}1] \\
        %     \vdots & \ddots & \vdots \\
        %     V[M{-}1] &\cdots & V[M^2{-}1] \\
        %     \end{array}\right]
        % = \bm{X} + b_1\bm{X}_{-1} + b_2\bm{X}_{-2} \\
        % &{=} \bm{X} + b_1\left[\begin{array}{c|c|c}
        %     V_0 & \cdots & V_{M^2{-}M} \\ 
        %     V_1 & \cdots & V_{M^2{-}M{+}1} \\
        %     \vdots & \ddots & \vdots \\
        %     V_{M{-}1} &\cdots & V_{M^2{-}1} \\
        %     \end{array}\right] +
        %     b_1\left[\begin{array}{c|c|c}
        %         V_0 & \cdots & V_{M^2{-}M} \\ 
        %         V_1 & \cdots & V_{M^2{-}M{+}1} \\
        %         \vdots & \ddots & \vdots \\
        %         V_{M{-}1} &\cdots & V_{M^2{-}1} \\
        %         \end{array}\right]
        \bm{W}^T &= \left[\begin{array}{c|c|c|c}
            w_0 & w_1 & \cdots & w_{M-1} \\ 
            w_M & w_{M+1} & \cdots & w_{2M-1} \\
            \vdots & \vdots & \ddots & \vdots \\
            w_{M^2-M} & w_{M^2-M+1} &\cdots & w_{M^2-1} \\
            \end{array}\right] \\
            &= \bm{V}^T + a_1\left[\begin{array}{c|c|c|c}
                0 & v_0 & v_1+a_1v_0 & \cdots \\ 
                0 & v_M & v_{M+1}+a_1v_M & \cdots \\
                \vdots & \vdots & \vdots & \ddots \\
                0 & v_{M^2-M} & v_{M^2-M+1}+a_1v_{M^2-M} & \cdots \\
                \end{array}\right]  \\
            &+ a_2\left[\begin{array}{c|c|c|c}
                0 & 0 & v_0 & \cdots \\ 
                0 & 0 & v_M & \cdots \\
                \vdots & \vdots & \ddots & \ddots \\
                0 & 0 & v_{M^2-M} & \cdots \\
                \end{array}\right]  \\
            &= \bm{V}^T + a_1\left[\begin{array}{c|c|c|c|c}
                0 & w_0 & w_1 & \cdots & w_{M-2} \\ 
                0 & w_M & w_{M+1} & \cdots & w_{2M-2} \\
                \vdots & \vdots & \vdots & \ddots & \vdots \\
                0 & w_{M^2-M} & w_{M^2-M+1} & \cdots & w_{M^2-2} \\
                \end{array}\right]  \\
            &+ a_2\left[\begin{array}{c|c|c|c|c}
                0 & 0 & w_0 & \cdots & w_{M-3} \\ 
                0 & 0 & w_M & \cdots & w_{2M-3} \\
                \vdots & \vdots & \ddots & \ddots & \ddots \\
                0 & 0 & w_{M^2-M} & \cdots & w_{M^2-3} \\
                \end{array}\right]  \\
    \end{aligned}
\end{equation}
where $\bm{W}^T$ is the transposed output matrix of ZIC. Note, \eqref{eq:ZIC_w_trans} doesn't compute
the complete particular solution of \eqref{eq:iir_filter_with_first_samples}. For example, for each sample in the first block, the compensations (multiplications by $a_1$ and $a_2$) are two empty blocks.
This means the samples except for $w_0$ will lack of some information of past few samples due to the recursive property, e.g.,
the complete solution of $w_m$ should be a combination of $v_0$, $v_1$, $\cdots$, $v_m$. Thus, when we do
initial condition correction (ICC), each block should be corrected based on the last two output samples of the previous
block instead of the original states $y_{-1}$ and $y_{-2}$ only. We will explain more details later in section ICC.
The reason for doing \eqref{eq:ZIC_w_trans} is to leverage the intra-block parallelism.

Also note, because of the intrinsic dependency problem of IIR filter, inter-block parallelism can't be achieved simultaneously
when process \eqref{eq:ZIC_w_trans}. The execution procedure is as follows:

1. Load and broadcast two coefficients $a_1$ and $a_2$ in SIMD vector.

2. load the first block of $\bm{V}^T$ and store it as the first block in the output matrix $\bm{W}^T$.

3. load the second block of $\bm{V}^T$ and apply one FMA with the previous output block to get the second block of $\bm{W}^T$. 

4. load the rest blocks for $M{-}2$ times and apply two FMAs with the previous two output blocks to get the entire $\bm{W}^T$.

\subsubsection{Non-transposed zero initial condition}
If we re-write \eqref{eq:ZIC_w_trans} with non-transposed data version, i.e.,

\begin{equation}
    \label{eq:ZIC_wo_trans}
    \begin{aligned}
        % \bm{V} &{=} \left[\begin{array}{c|c|c}
        %     V[0] & \cdots & V[M^2{-}M] \\ 
        %     V[1] & \cdots & V[M^2{-}M{+}1] \\
        %     \vdots & \ddots & \vdots \\
        %     V[M{-}1] &\cdots & V[M^2{-}1] \\
        %     \end{array}\right]
        % = \bm{X} + b_1\bm{X}_{-1} + b_2\bm{X}_{-2} \\
        % &{=} \bm{X} + b_1\left[\begin{array}{c|c|c}
        %     V_0 & \cdots & V_{M^2{-}M} \\ 
        %     V_1 & \cdots & V_{M^2{-}M{+}1} \\
        %     \vdots & \ddots & \vdots \\
        %     V_{M{-}1} &\cdots & V_{M^2{-}1} \\
        %     \end{array}\right] +
        %     b_1\left[\begin{array}{c|c|c}
        %         V_0 & \cdots & V_{M^2{-}M} \\ 
        %         V_1 & \cdots & V_{M^2{-}M{+}1} \\
        %         \vdots & \ddots & \vdots \\
        %         V_{M{-}1} &\cdots & V_{M^2{-}1} \\
        %         \end{array}\right]
        &\bm{W} = \left[\begin{array}{c|c|c|c}
            w_0 & w_M & \cdots & w_{M^2-M} \\ 
            w_1 & w_{M+1} & \cdots & w_{M^2-M+1} \\
            \vdots & \vdots & \ddots & \vdots \\
            w_{M-1} & w_{2M-1} &\cdots & w_{M^2-1} \\
            \end{array}\right] = \bm{V} \\
            &+ a_1\left[\begin{array}{c|c|c|c}
                0 & 0 & \cdots & 0 \\ 
                v_0 & v_M & \cdots & v_{M^2-M} \\
                v_1{+}a_1v_0 & v_{M+1}{+}a_1v_M & \cdots & v_{M^2-M+1}{+}a_1v_{M^2-M} \\
                \vdots & \vdots & \ddots & \vdots \\
                \end{array}\right]  \\
            &+ a_2\left[\begin{array}{c|c|c|c}
                0 & 0 & \cdots & 0 \\ 
                0 & 0 & \cdots & 0 \\
                v_0 & v_M & \cdots & v_{M^2-M} \\
                \vdots & \vdots & \ddots & \vdots \\
                \end{array}\right]  \\
    \end{aligned}
\end{equation}
We can see that \eqref{eq:ZIC_wo_trans} can no longer be computed among multiple blocks. This is because
the dependency problem now occurs inside each block instead of happening among blocks as \eqref{eq:ZIC_w_trans}.
However, \eqref{eq:ZIC_wo_trans} can still achieve intra-block parallelism by block-filtering algorithm based on \eqref{eq:iir_filter_with_first_samples} as shown below

\begin{equation}
    \label{eq:vector_ZIC_wo_trans}
    \begin{aligned}
        % \bm{V} &{=} \left[\begin{array}{c|c|c}
        %     V[0] & \cdots & V[M^2{-}M] \\ 
        %     V[1] & \cdots & V[M^2{-}M{+}1] \\
        %     \vdots & \ddots & \vdots \\
        %     V[M{-}1] &\cdots & V[M^2{-}1] \\
        %     \end{array}\right]
        % = \bm{X} + b_1\bm{X}_{-1} + b_2\bm{X}_{-2} \\
        % &{=} \bm{X} + b_1\left[\begin{array}{c|c|c}
        %     V_0 & \cdots & V_{M^2{-}M} \\ 
        %     V_1 & \cdots & V_{M^2{-}M{+}1} \\
        %     \vdots & \ddots & \vdots \\
        %     V_{M{-}1} &\cdots & V_{M^2{-}1} \\
        %     \end{array}\right] +
        %     b_1\left[\begin{array}{c|c|c}
        %         V_0 & \cdots & V_{M^2{-}M} \\ 
        %         V_1 & \cdots & V_{M^2{-}M{+}1} \\
        %         \vdots & \ddots & \vdots \\
        %         V_{M{-}1} &\cdots & V_{M^2{-}1} \\
        %         \end{array}\right]
        \bm{W} &= \left[\begin{array}{c|c|c|c}
            w_0 & w_M & \cdots & w_{M^2-M} \\ 
            w_1 & w_{M+1} & \cdots & w_{M^2-M+1} \\
            \vdots & \vdots & \ddots & \vdots \\
            w_{M-1} & w_{2M-1} &\cdots & w_{M^2-1} \\
            \end{array}\right]  \\
            &= \left[\begin{array}{ccccc}
                1 & 0 & \cdots & \cdots & 0 \\ 
                a_1 & 1 & \ddots & 0  & \vdots \\
                a_1^2+a_2 & a_1 & 1 & \ddots & \vdots \\
                a_1^3+2a_1a_2 & a_1^2+a_2 & a_1 & 1 & 0 \\
                \vdots & \vdots & \vdots & \vdots & \ddots\vdots \\
                \end{array}\right] \bm{V} = \bm{H}\bm{V} \\  
    \end{aligned}
\end{equation}
where \eqref{eq:vector_ZIC_wo_trans} performs a matrix multiplication and $\bm{H}$ denotes
the transition matrix from $\bm{V}$ to $\bm{W}$. $\bm{H}$ can be pre-computed
by an impulse response, $\left[0 ~ 1~ \dashline ~ a_1 ~ a_1^2+a_2 ~ (a_1^2+a_2)a_1{+}a_1a_2~\cdots\right]$,
where each response is obtained by multiplying the previous two responses with $a_2$ and $a_1$.
The vertical dashed line is used to seperate the initial state and the resulting response sequence,
which is denoted by $\bm{h}_1 = \left[a_1 ~ a_1^2+a_2 ~ (a_1^2+a_2)a_1{+}a_1a_2~\cdots\right]$.
The length of $\bm{h}_1$ is $M$.
The execution procedure of \eqref{eq:vector_ZIC_wo_trans} is as follows:

1. Pre-calculate and load the transition matrix $\bm{H}$. Note, this won't take the run time.

2. Load and broadcast every sample in the first block of $\bm{V}$. Apply $M$ times FMAs with each SIMD vectors at the same index in $\bm{H}$
and accumulate them together to get the first block of $\bm{W}$.

3. Repeat step 2 for $M{-}1$ times for the rest blocks to get the entire $\bm{W}$.

Note, the resulting $\bm{W}$ from \eqref{eq:vector_ZIC_wo_trans} is exactly the same as 
$\bm{W}^T$ from \eqref{eq:ZIC_w_trans} without matrix transpose.

\subsubsection{Non-transposed initial condition correction (ICC)}

After computing the particular solution of second order IIR filter, we now compensate for the 
initial conditions to get the complete solution. Recall that the output matrix of ZIC in either \eqref{eq:ZIC_w_trans} or
\eqref{eq:vector_ZIC_wo_trans} only has partial particular solution. Thus, based on \eqref{eq:iir_filter_with_first_samples} and along with ZIC,
the algorithm that computes the homogeneous solution of second order
IIR filter with non-transposed matrix of data is given by 

\begin{equation}
    \label{eq:ICC_wo_trans}
    \begin{aligned}
        % \bm{V} &{=} \left[\begin{array}{c|c|c}
        %     V[0] & \cdots & V[M^2{-}M] \\ 
        %     V[1] & \cdots & V[M^2{-}M{+}1] \\
        %     \vdots & \ddots & \vdots \\
        %     V[M{-}1] &\cdots & V[M^2{-}1] \\
        %     \end{array}\right]
        % = \bm{X} + b_1\bm{X}_{-1} + b_2\bm{X}_{-2} \\
        % &{=} \bm{X} + b_1\left[\begin{array}{c|c|c}
        %     V_0 & \cdots & V_{M^2{-}M} \\ 
        %     V_1 & \cdots & V_{M^2{-}M{+}1} \\
        %     \vdots & \ddots & \vdots \\
        %     V_{M{-}1} &\cdots & V_{M^2{-}1} \\
        %     \end{array}\right] +
        %     b_1\left[\begin{array}{c|c|c}
        %         V_0 & \cdots & V_{M^2{-}M} \\ 
        %         V_1 & \cdots & V_{M^2{-}M{+}1} \\
        %         \vdots & \ddots & \vdots \\
        %         V_{M{-}1} &\cdots & V_{M^2{-}1} \\
        %         \end{array}\right]
        &\bm{Y} = \left[\begin{array}{c|c|c|c}
            y_0 & y_M & \cdots & y_{M^2-M} \\ 
            y_1 & y_{M+1} & \cdots & y_{M^2-M+1} \\
            \vdots & \vdots & \ddots & \vdots \\
            y_{M-1} & y_{2M-1} &\cdots & y_{M^2-1} \\
            \end{array}\right] = \bm{W} + \\
            &\left[\begin{array}{cc}
                a_1 & a_2 \\ 
                a_1^2+a_2 & a_1a_2 \\
                a_1^3+2a_1a_2 & a_1^2a_2+a_2^2 \\
                \vdots & \vdots
                \end{array}\right]  
                \left[\begin{array}{cccc}
                    y_{-1} & y_{M-1} & \cdots & y_{M^2-M-1} \\ 
                    y_{-2} & y_{M-2} & \cdots & y_{M^2-M-2} \\
                    \end{array}\right] \\
            &= \bm{W} + \bm{h}\bm{Y}_p
    \end{aligned}
\end{equation}
where $\bm{h}$ denotes the transition matrix for the pre-state matrix $\bm{Y}_p$, and it is composed by
two impulse response sequences, $\bm{h}_1$ and $\bm{h}_2$, where $\bm{h}_2$ is the resulting response sequence of 
$\left[1 ~ 0~ \dashline ~ a_2 ~ a_1a_2 ~ a_1^2a_2{+}a_2^2~\cdots\right]$,
following the same rule as $\bm{h}_1$. Compared with \eqref{eq:iir_filter_with_first_samples}, because the
output samples of ZIC from \eqref{eq:ZIC_w_trans} or \eqref{eq:vector_ZIC_wo_trans} only has partial particular solution,
we can't compensate for every block by the original initial conditions $y_{-1}$ and $y_{-2}$.
Instead, we correct each block based on the pre-states of the start sample at each block, e.g., $\bm{W}[1]$ should
be compensated by $\bm{Y}[0][M-1]$ and $\bm{Y}[0][M-2]$. Furthermore,
\eqref{eq:ICC_wo_trans} exploits intra-block parallelism and multi-block filtering method. 
The execution procedure is as follows:

1. Pre-calculate and load the transition matrix $\bm{h}$.

2. Load and broadcast the first two initial conditions $y_{-1}$, $y_{-2}$ and apply two FMAs with two SIMD vectors 
in $\bm{h}$ to get the first block of $\bm{Y}$.

3. Repeat step 2 for $M{-}1$ times with two initial conditions, which are the last two samples of the prvious output block,
to get the entire $\bm{Y}$.

4. Store the last block of $\bm{Y}$ in the shift register as the initial
conditions for next IIR filtering.

\subsubsection{Transposed initial condition correction (ICC)}

To compute the homogeneous solution with transposed matrix of data, we re-write \eqref{eq:ICC_wo_trans}
in transposed version, i.e.,

\begin{equation}
    \label{eq:ICC_w_trans}
    \begin{aligned}
        % \bm{V} &{=} \left[\begin{array}{c|c|c}
        %     V[0] & \cdots & V[M^2{-}M] \\ 
        %     V[1] & \cdots & V[M^2{-}M{+}1] \\
        %     \vdots & \ddots & \vdots \\
        %     V[M{-}1] &\cdots & V[M^2{-}1] \\
        %     \end{array}\right]
        % = \bm{X} + b_1\bm{X}_{-1} + b_2\bm{X}_{-2} \\
        % &{=} \bm{X} + b_1\left[\begin{array}{c|c|c}
        %     V_0 & \cdots & V_{M^2{-}M} \\ 
        %     V_1 & \cdots & V_{M^2{-}M{+}1} \\
        %     \vdots & \ddots & \vdots \\
        %     V_{M{-}1} &\cdots & V_{M^2{-}1} \\
        %     \end{array}\right] +
        %     b_1\left[\begin{array}{c|c|c}
        %         V_0 & \cdots & V_{M^2{-}M} \\ 
        %         V_1 & \cdots & V_{M^2{-}M{+}1} \\
        %         \vdots & \ddots & \vdots \\
        %         V_{M{-}1} &\cdots & V_{M^2{-}1} \\
        %         \end{array}\right]
        &\bm{Y}^T = \left[\begin{array}{c|c|c|c}
            y_0 & y_1 & \cdots & y_{M-1} \\ 
            y_M & y_{M+1} & \cdots & y_{2M-1} \\
            \vdots & \vdots & \ddots & \vdots \\
            y_{M^2-M} & y_{M^2-M+1} &\cdots & y_{M^2-1} \\
            \end{array}\right] = \bm{W}^T + \\
            &\left[\begin{array}{c|c}
                y_{-1} & y_{-2} \\ 
                y_{M-1} & y_{M-2} \\
                \vdots & \vdots \\
                y_{M^2-M-1} & y_{M^2-M-2}
                \end{array}\right]  
                \left[\begin{array}{cccc}
                    a_1 & a_1^2+a_2 & a_1^3+2a_1a_2 & \cdots \\ 
                    a_2 & a_1a_2 & a_1^2a_2+a_2^2 & \cdots \\
                    \end{array}\right] \\
            &= \bm{W}^T + \bm{Y}_p^T\bm{h}^T
    \end{aligned}
\end{equation}
Compared with \eqref{eq:ICC_wo_trans}, we need one extra step to process \eqref{eq:ICC_w_trans} 
that is to compute the two initial-condition vectors in $\bm{Y}_p^T$. The two initial conditions
can be computed at first since they only depend on $\bm{W}$ and the two pre-states $y_{-2}$ and $y_{-1}$.
Thus, the algorithm of calculating $\bm{Y}_p^T$ is given by

\begin{equation}
    \label{eq:ICC_init_w_trans}
    \begin{aligned}
        % \bm{V} &{=} \left[\begin{array}{c|c|c}
        %     V[0] & \cdots & V[M^2{-}M] \\ 
        %     V[1] & \cdots & V[M^2{-}M{+}1] \\
        %     \vdots & \ddots & \vdots \\
        %     V[M{-}1] &\cdots & V[M^2{-}1] \\
        %     \end{array}\right]
        % = \bm{X} + b_1\bm{X}_{-1} + b_2\bm{X}_{-2} \\
        % &{=} \bm{X} + b_1\left[\begin{array}{c|c|c}
        %     V_0 & \cdots & V_{M^2{-}M} \\ 
        %     V_1 & \cdots & V_{M^2{-}M{+}1} \\
        %     \vdots & \ddots & \vdots \\
        %     V_{M{-}1} &\cdots & V_{M^2{-}1} \\
        %     \end{array}\right] +
        %     b_1\left[\begin{array}{c|c|c}
        %         V_0 & \cdots & V_{M^2{-}M} \\ 
        %         V_1 & \cdots & V_{M^2{-}M{+}1} \\
        %         \vdots & \ddots & \vdots \\
        %         V_{M{-}1} &\cdots & V_{M^2{-}1} \\
        %         \end{array}\right]
        &\left[\begin{array}{c}
            y_{M-2}  \\ 
            y_{2M-2}  \\
            \vdots \\
            y_{M^2-2} \\ \hline 
            y_{M-1}  \\ 
            y_{2M-1}  \\
            \vdots \\
            y_{M^2-1} \\
            \end{array}\right] = 
            \left[\begin{array}{cc}
                \bm{h}_{2,M-2} & \bm{h}_{1,M-2} \\ 
                \bm{h}_{2,M-2}\bm{h}_{2,M-2} & \bm{h}_{1,M-2}\bm{h}_{2,M-2} \\
                +\bm{h}_{2,M-1}\bm{h}_{1,M-2} & +\bm{h}_{1,M-1}\bm{h}_{1,M-2} \\
                \vdots & \vdots \\ \hline
                \bm{h}_{2,M-1} & \bm{h}_{1,M-1} \\ 
                \bm{h}_{2,M-2}\bm{h}_{2,M-1} & \bm{h}_{1,M-2}\bm{h}_{2,M-1} \\
                +\bm{h}_{2,M-1}\bm{h}_{1,M-1} & +\bm{h}_{1,M-1}\bm{h}_{1,M-1} \\
                \vdots & \vdots
                \end{array}\right]
                \left[\begin{array}{c}
                    y_{-2}  \\ 
                    y_{-1}  \\
                    \end{array}\right] + \\
            &\left[\begin{array}{ccc|ccc}
                1 & 0 & \cdots & 0 & \cdots & \cdots \\ 
                \bm{h}_{2,M-2} & 1 & 0 & \bm{h}_{1,M-2} & 0 & 0 \\
                % \bm{h}_2[M{-}2]\bm{h}_2[M{-}2]+\bm{h}_2[M{-}1]\bm{h}_1[M{-}2]
                \bm{h}_{2,M-2}\bm{h}_{2,M-2}+ & \ddots & \ddots\vdots & \bm{h}_{1,M-2}\bm{h}_{2,M-2}+ & \ddots & \ddots\vdots \\
                \bm{h}_{2,M-1}\bm{h}_{1,M-2}  & & & \bm{h}_{1,M-1}\bm{h}_{1,M-2} \\
                \vdots & \vdots & \ddots\vdots & \vdots & \vdots & \ddots\vdots \\ \hline
                0 & \cdots & \cdots & 1 & 0 & \cdots \\ 
                \bm{h}_{2,M-1} & 0 & 0 & \bm{h}_{1,M-1} & 1 & 0 \\
                % \bm{h}_2[M{-}2]\bm{h}_2[M{-}2]+\bm{h}_2[M{-}1]\bm{h}_1[M{-}2]
                \bm{h}_{2,M-2}\bm{h}_{2,M-1}+ & \ddots & \ddots\vdots & \bm{h}_{1,M-2}\bm{h}_{2,M-1}+ & \ddots & \ddots\vdots \\
                \bm{h}_{2,M-1}\bm{h}_{1,M-1}  & & & \bm{h}_{1,M-1}\bm{h}_{1,M-1} \\
                \vdots & \vdots & \ddots\vdots & \vdots & \vdots & \ddots\vdots \\      
            \end{array}\right] \\ 
            &\cdot \left[\begin{array}{cccc|cccc}
                w_{M-2} & w_{2M-2} & \cdots & w_{M^2-2} & w_{M-1} &  & \cdots & w_{M^2-1} \\ 
                \end{array}\right]^T \\
            &= \bm{D}\bm{y}_p + \bm{T} \cdot 
            \left[\begin{array}{c}
                \bm{W}^T[M{-}2]  \\ 
                \bm{W}^T[M{-}1]  \\
                \end{array}\right]
            % &\left[\begin{array}{c|c}
            %     a_1 & a_2 \\ 
            %     a_1^2+a_2 & a_1a_2 \\
            %     a_1^3+2a_1a_2 & a_1^2a_2+a_2^2 \\
            %     \vdots & \vdots
            %     \end{array}\right]  
            %     \left[\begin{array}{cccc}
            %         y_{-1} & y_{M-1} & \cdots & y_{M^2-M-1} \\ 
            %         y_{-2} & y_{M-2} & \cdots & y_{M^2-M-2} \\
            %         \end{array}\right] \\
            % &= \bm{W} + \bm{h}\bm{Y}_p
    \end{aligned}
\end{equation}
where $\bm{h}_{x,n}$ is equivalent to $\bm{h}_x[n]$ and 
$\bm{D}$ and $\bm{T}$ are two transition matrices for the pre-state vector $\bm{y}_p$
and the corresponding part of $\bm{W}$. 
$\bm{D}$ and $\bm{T}$ can be pre-computed by the following two joint
impulse responses, which are 

1. $\Big[1 ~ 0~ \dashline ~\bm{h}_{2,M-2} ~\bm{h}_{2,M-1} ~\bm{h}_{2,M-2}\bm{h}_{2,M-2}{+}\bm{h}_{2,M-1}\bm{h}_{1,M-2} $

$\quad\quad\quad\quad\quad\quad\quad\quad\quad\quad ~ \bm{h}_{2,M-2}\bm{h}_{2,M-1}{+}\bm{h}_{2,M-1}\bm{h}_{1,M-1} \cdots\Big]$

2. $\Big[0 ~ 1~ \dashline ~\bm{h}_{1,M-2} ~\bm{h}_{1,M-1} ~\bm{h}_{1,M-2}\bm{h}_{2,M-2}{+}\bm{h}_{1,M-1}\bm{h}_{1,M-2} $

$\quad\quad\quad\quad\quad\quad\quad\quad\quad\quad ~ \bm{h}_{1,M-2}\bm{h}_{2,M-1}{+}\bm{h}_{1,M-1}\bm{h}_{1,M-1} \cdots\Big]$
The first position of initial state corresponds to $\bm{h}_2$ while the second one points to $\bm{h}_1$. Furthermore,
each of the above two impulse sequences calculates a pair of impulse responses simultaneously and interleavedly.
The current pair of impulse response are calculated based on the previous pair and the former determines multiplying $\bm{h}_x$ of position $M{-}2$  
while the latter points to $M{-}1$. 

Note, after computing \eqref{eq:ICC_init_w_trans}, the pre-state matrix $\bm{Y}_p^T$ in \eqref{eq:ICC_w_trans}
can be then obtained by shuffling with $y_{-1}$ and $y_{-2}$. Furthermore,
since $\bm{h}^T$ can be pre-computed,
the process of computing \eqref{eq:ICC_w_trans} achieves both intra-block and inter-block parallelism.
The execution procedure of \eqref{eq:ICC_init_w_trans} and \eqref{eq:ICC_w_trans} is as follows:

1. Pre-compute and load two transition matrices $\bm{D}$ and $\bm{T}$.

2. Load and broadcast $y_{-1}$, $y_{-2}$ and each sample in the last two blocks of $\bm{W}$, then apply
$4M{+}4$ FMAs to get the last two blocks of $\bm{Y}^T$ following \eqref{eq:ICC_init_w_trans}.

3. Shuffle the obtained two output blocks at step 2 with $y_{-1}$ and $y_{-2}$ respectively to get the pre-state matrix $\bm{Y}_p^T$.

4. Load and broadcast each element in $\bm{h}^T$ and apply two FMAs for $M{-}2$ times to get the entire $\bm{Y}^T$ following \eqref{eq:ICC_w_trans}.

\subsection{Matrix transpose}


Since the proposed algorithm accepts both transposed and non-transposed data at each processing phase, it is necessary to discuss
how matrix transpose is performed in SIMD operation. In this paper, the matrix transpose for multiple blocks of data is executed as follows:

1. Swap lower left part and upper right of the matrix, i.e.,

\begin{equation*}
    \begin{aligned}
    \label{eq:matrix_trans_step_1}
    % \bm{X} = \left[\begin{array}{c|c|c|c}
    % x[0] & x[M] & \cdots & x[M^2{-}M] \\ 
    % x[1] & x[M{+}1] & \cdots & x[M^2{-}M{+}1] \\
    % \vdots & \vdots & \ddots & \vdots \\
    % x[M{-}1] & x[2M{-}1] &\cdots & x[M^2{-}1] \\
    % \end{array}\right]
    &\bm{X} \rightarrow \\
    &\left[\begin{array}{c|c|c|c|c}
        x_0 & x_M & & x_{M/2} &  \\ 
        x_1 & x_{M+1} & & x_{M/2+1} & \\
        \vdots & \vdots & & \vdots & \\
        x_{M/2-1} & x_{M+M/2-1} & \cdot\cdot & x_{M-1} & \cdot\cdot \\
        x_{M/2\cdot M} & x_{(M/2+1)\cdot M} & & x_{M/2\cdot M+M/2} & \\
        x_{M/2\cdot M+1} & x_{(M/2+1)\cdot M+1} & & x_{M/2\cdot M+M/2+1} & \\
        \vdots & \vdots & & \vdots & \\
        x_{M/2\cdot (M+1)-1} & x_{(M/2+1)\cdot M+M/2-1} & & x_{(M/2+1)M-1} & \\
        \end{array}\right]
    \end{aligned}
\end{equation*}

2. Swap lower left part and upper right within each quadrant,

\begin{equation*}
    \label{eq:matrix_trans_step_2}
    % \bm{X} = \left[\begin{array}{c|c|c|c}
    % x[0] & x[M] & \cdots & x[M^2{-}M] \\ 
    % x[1] & x[M{+}1] & \cdots & x[M^2{-}M{+}1] \\
    % \vdots & \vdots & \ddots & \vdots \\
    % x[M{-}1] & x[2M{-}1] &\cdots & x[M^2{-}1] \\
    % \end{array}\right]
    \rightarrow 
    \left[\begin{array}{c|c|c|c}
        x_0 & x_M & x_2 &  \\ 
        \vdots & \vdots & \vdots & \\
        x_{M/4-1} & x_{M+M/4-1} & x_{M/2-1} & \\
        x_{M/4\cdot M} & x_{(M/4+1)\cdot M} & x_{M/4\cdot M+2} &  \\
        \vdots & \vdots & \vdots & \\
        x_{M/4\cdot M+M/4-1} & x_{M/4\cdot M+5M/4-1} & x_{M/4\cdot M+M/2-1} & \cdot\cdot \\
        x_{M/2\cdot M} & x_{(M/2+1)\cdot M} & x_{M/2\cdot M+2} &  \\ 
        \vdots & \vdots & \vdots & \\
        x_{M/2\cdot M+M/4-1} & x_{M/2\cdot M+5M/4-1} & x_{M/2\cdot M+M/2-1} & \\
        x_{3M/2\cdot M} & x_{(3M/2+1)\cdot M} & x_{3M/2\cdot M+2} &  \\
        \vdots & \vdots & \vdots & \\
        x_{3M/2\cdot M+M/4-1} & x_{3M/2\cdot M+5M/4-1} & x_{3M/2\cdot M+M/2-1} & \\    
    \end{array}\right]
\end{equation*}

3. Repeat step 2 with each sub-quadrant until step 4 is ready

4. Swap anti-diagonally adjacent elements to get $\bm{X}^T$.

It can be seen that the entire process of matrix transpose in SIMD operation requires $M\log(M)$
shuffles.


\section{Implementation of the Proposed Algorithm}
\label{sec:implementation}

Having introduced the solutions of a second order recursive filter, now we discuss and compare with different
combinations of functions to build the second or higher order recursive filter from implementation perspective.
% Assume the input is a matrix of consecutive samples. Thus, for block filtering, the functions for solving particular and homogeneous
% parts should perform $M$ times.

The option 1 only accepts non-transposed matrix of samples, i.e., the block filtering method, which can be considered as the benchmark.
The option 2 is that the particular part of recursive equation is solved by multi-block filtering, 
and block filtering is working for the homogeneous part, which basically has the same idea as multi-block filtering method in \cite{Jaewoo_09}.
The option 3 is our proposed multi-block filtering algorithm, where both functions for solving particular and homogeneous parts accept transposed matrix of samples.
From implementation aspects, we are not limited to count the number of ALUs (FMAs, fused multiplication and additions), but the number of other SIMD instructions, 
e.g., shuffle, broadcast and load operations for the basic functions and options, as shown in Table \ref{table:number_of_instructions}.
The number of store operation for each function is identical as $1/M$. 

\begin{table}[t]
    \caption{The number of SIMD instructions (per sample) for functions and options in theory       }  % title of Table
    \centering % used for centering table
    \setlength{\tabcolsep}{0.9pt}
    \begin{tabular}{c|c|c|c|c} % centered columns (4 columns)
    \hline\hline %inserts double horizontal lines
    function name & FMA & shuffle & broadcast & load \\ [1ex] % inserts table
    \hline % inserts single horizontal line
    NT ZIC & $1{+}\frac{2}{M}$ & 0 & $1{+}\frac{2}{M}$ & $1{+}\frac{5}{M}$ \\ [0.3ex]
    NT ICC & $\frac{2}{M}$ & 0 & $\frac{2}{M}$ & $\frac{5}{M}$ \\ [0.3ex]
    T ZIC & $\frac{4}{M}{-}\frac{3}{M^2}$ & $\frac{2}{M^2}$ & $\frac{4}{M^2}$ & $\frac{1}{M}{+}\frac{6}{M^2}$ \\ [0.3ex]
    T ICC & $\frac{4\log_2M}{M^2}{+}\frac{2}{M}$ & $\frac{2\log_2M}{M^2}{+}\frac{2}{M^2}$ & $\frac{2}{M}{-}\frac{2}{M^2}$ & $\frac{1}{M}{+}\frac{4\log_2M{+}8}{M^2}$ \\ [0.3ex]
    Transpose & 0 & $\frac{\log_2M}{M}$ & 0 & $\frac{1}{M}$ \\ [0.3ex]
    Option 1 & $1{+}\frac{4}{M}$ & 0 & $1{+}\frac{4}{M}$ & $1{+}\frac{9}{M}$ \\ [0.3ex]
    Option 2 & $\frac{6}{M}{-}\frac{3}{M^2}$ & $\frac{2\log_2M}{M}{+}\frac{2}{M^2}$ & $\frac{2}{M}{+}\frac{4}{M^2}$ & $\frac{1}{M}{+}\frac{10}{M^2}$ \\ [0.3ex]
    Option 3 & $\frac{6}{M}{+}\frac{4\log_2M{-}3}{M^2}$ & $\frac{2\log_2M}{M}{+}\frac{2\log_2M+4}{M^2}$ & $\frac{2}{M}{+}\frac{2}{M^2}$ & $\frac{1}{M}{+}\frac{4\log_2M{+}14}{M^2}$ \\ [1ex]
    \hline
    \end{tabular}
    \label{table:number_of_instructions} % is used to refer this table in the text
\end{table}

From Table \ref{table:number_of_instructions}, we basically see that the algorithms are scaled linearly by the length of SIMD instructions,
i.e., the number of instructions per cycle is inversely propotional to $M$ or $M^2$. 
NT ZIC (or NT ICC) denotes the function of block filtering method for solving the particular (or homogeneous) part of recursive equation,
where NT stands for non-transposed grid of samples (the samples are processed consecutively in vector) to distinguish from transposed grid of samples
in multi-block filtering. 
Assume we implement the algorithms on software. Normally, the load and broadcast operation are cheap, and moreover,
are known to the compiler at the compile time. The smart compiler will interleave the two operations by FMA and shuffles to save the cycles. 
On the other hand, if we re-compare each function by the number of FMA and shuffles, we may get similar results as before since the complexity degree for basic functions don't change.

Let's go beyond the function level to see the three options of compusing the second order filter. Option 1 is more useful when the processor is limited to deal with the SIMD instruction.
For modern CPUs, option 2 is the cheapest at the number of FMAs and shuffles. The feature of option 3 is that the two matrix transposes
are located at the head and tail (one for input to ZIC, one for getting correct order of samples). When a high order recursive filter is constructed by cascading second order sections,
the matrix transpose at the head of current section is cancelled out with the matrix transpose at the tail of previous section. 
Thus, each intermediate cascaded second order filters of option 3 only needs to take ZIC and ICC.
The complexity of intermediate sections of option 3 is $O(\log_2M/M^2)$, which is smaller than the complexity $O(\log_2M/M)$
of option 2.

Now we can see that our proposed algorithm is very suitable for building cascaded form of high order recursive filter.
If the number of instructions is only the consideration, then for a $L$th order recursive filter,
the cheapest combination can be $L/2-1$ cascaded second order filters of option 3 (without intermediate matrix transpose)
followed by 1 second order section of option 2.

However, there is one last problem at ICC part of option 2. In block filtering algorithm, the computation from each vector
relies on the last two samples of previous output vector. Thus, 
NT ICC may suffer from a strong dependency problem, which degrades the efficiency,
that can't be reflected from the number of instructions.
The solution is, instead of just looking at the number of instructions,  
we should go deeper to see the clock cycles of the functions and options.





% \subsection{Combination of a second (or higher) order IIR core}

% \subsubsection{The second order IIR core}

% \begin{table}[t]
%     \caption{The number of instructions for base functions in IIR filter}  % title of Table
%     \centering % used for centering table
%     \begin{tabular}{c c c c c c} % centered columns (4 columns)
%     \hline\hline %inserts double horizontal lines
%     function name & FMA & shuffle & broadcast & load & store \\ [1ex] % inserts table
%     \hline % inserts single horizontal line
%     Transposed FIR & $2M$ & 4 & 2 & $M{+}2$ & $M$ \\ [0.3ex]
%     Non-trans FIR & $2M$ & $2M$ & 2 & $M{+}2$ & $M{+}1$ \\ [0.3ex]
%     Transposed ZIC & $2M{-}3$ & 0 & 2 & $M{+}2$ & $M$ \\ [0.3ex]
%     Non-trans ZIC & $M^2$ & 0 & $M^2$ & $2M$ & $M$ \\ [0.3ex]
%     Transposed ICC & $6M$ & 4 & $4M{+}6$ & $5M{+}6$ & $M{+}4$ \\ [0.3ex]
%     Non-trans ICC & $2M$ & 0 & $2M$ & $M{+}2$ & $M$ \\ [0.3ex]
%     Matrix transpose & 0 & $M{\log_2}M$ & 0 & $M$ & $M$ \\ [1ex]
%     \hline
%     \end{tabular}
%     \label{table:number_of_instructions} % is used to refer this table in the text
% \end{table}

% In the previous section, lots of base functions of composing a second order IIR core by passing transposed or non-transposed data are introduced.
% We now count for the number of FMAs, shuffles, broadcast, load and store operations for each base function
% as shown in Table \ref{table:number_of_instructions}. Based on the number of instructions (FMAs and shuffles) for each function,
% we list three significant combinations of a second order IIR filter and compute the total number of instructions as follows: 

% 1. Core 1: all passed non-transposed data. 
% \begin{equation*}
%     \label{eq:iir_core_1_all_non}
%     % \bm{X} = \left[\begin{array}{c|c|c|c}
%     % x[0] & x[M] & \cdots & x[M^2{-}M] \\ 
%     % x[1] & x[M{+}1] & \cdots & x[M^2{-}M{+}1] \\
%     % \vdots & \vdots & \ddots & \vdots \
%     % x[M{-}1] & x[2M{-}1] &\cdots & x[M^2{-}1] \\
%     % \end{array}\right]
%         \begin{array}{c|c|c|c|c|c|c}
%             \cline{2-2}
%             \cline{4-4}
%             \cline{6-6}
%             \underrightarrow{\bm{X}} & \text{FIR} & \underrightarrow{\bm{V}} & \text{ZIC} & \underrightarrow{\bm{W}} & \text{ICC} & \underrightarrow{\bm{Y}} \\
%             \cline{2-2}
%             \cline{4-4}
%             \cline{6-6}
%     \end{array}
% \end{equation*}

% 2. Core 2: all passed transposed data.
% \begin{equation*}
%     \label{eq:iir_core_2_all_trans}
%     \begin{aligned}
%     % \bm{X} = \left[\begin{array}{c|c|c|c}
%     % x[0] & x[M] & \cdots & x[M^2{-}M] \\ 
%     % x[1] & x[M{+}1] & \cdots & x[M^2{-}M{+}1] \\
%     % \vdots & \vdots & \ddots & \vdots \\
%     % x[M{-}1] & x[2M{-}1] &\cdots & x[M^2{-}1] \\
%     % \end{array}\right]
%     & \underrightarrow{\bm{X}} \\
%         & \begin{array}{|c|c|c|c|c|c|c|c|c|c}
%             \cline{1-1}
%             \cline{3-3}
%             \cline{5-5}
%             \cline{7-7}
%             \cline{9-9}
%             \text{T} & \underrightarrow{\bm{X}^T} & \text{T FIR} & \underrightarrow{\bm{V}^T} & \text{T ZIC} & \underrightarrow{\bm{W}^T} & \text{T ICC} & \underrightarrow{\bm{Y}^T} & \text{T} & \underrightarrow{\bm{Y}} \\
%             \cline{1-1}
%             \cline{3-3}
%             \cline{5-5}
%             \cline{7-7}
%             \cline{9-9}
%     \end{array}
% \end{aligned}
% \end{equation*}

% 3. Core 3: only ICC passed non-transposed data.
% \begin{equation*}
%     \label{eq:iir_core_3_icc_non_trans}
%     \begin{aligned}
%     % \bm{X} = \left[\begin{array}{c|c|c|c}
%     % x[0] & x[M] & \cdots & x[M^2{-}M] \\ 
%     % x[1] & x[M{+}1] & \cdots & x[M^2{-}M{+}1] \\
%     % \vdots & \vdots & \ddots & \vdots \\
%     % x[M{-}1] & x[2M{-}1] &\cdots & x[M^2{-}1] \\
%     % \end{array}\right]
%     % & \underrightarrow{\bm{X}} \\
%         & \begin{array}{c|c|c|c|c|c|c|c|c|c|c}
%             \cline{2-2}
%             \cline{4-4}
%             \cline{6-6}
%             \cline{8-8}
%             \cline{10-10}
%             \underrightarrow{\bm{X}} & \text{T} & \underrightarrow{\bm{X}^T} & \text{T FIR} & \underrightarrow{\bm{V}^T} & \text{T ZIC} & \underrightarrow{\bm{W}^T} & \text{T} & \underrightarrow{\bm{W}} & \text{ICC} & \underrightarrow{\bm{Y}} \\
%             \cline{2-2}
%             \cline{4-4}
%             \cline{6-6}
%             \cline{8-8}
%             \cline{10-10}
%     \end{array}
% \end{aligned}
% \end{equation*}
% The total cost for three cores are: $M^2{+}6M$, $2M{\log_2}M{+}10M{+}5$, $2M{\log_2}M{+}6M{+}1$, respectively.
% Normally, the length of SIMD vector is short, e.g., 4 or 8. Let's assume $M=8$ to achieve the better
% parallelism. Thus, the cost for three cores are calculated as 112, 133 and 97.

% If we consider the basic situation, i.e., a single CPU and single thread are used, core 3 should be
% the most instruction-saving. The feature of core 2 is that the two matrix transpose functions exist at the tail and head.
% If we construct a higher order IIR filter by cascading multiple second order IIR cores, the two matrix transpose functions
% at the tail and head can be cancelled out. In this case, the cost for core 2 can be then reduced to $10M{+}5=85$, which is cheaper
% than core 3. The advantage of core 1 is straightforward thus avoids two matrix transpose functions.
% Furthermore, if we consider a multi-CPU and multi-thread environment, the most expensive function of core 1, i.e., non-transposed ZIC,
% can be executed by inter-block parallelism and the real number of instructions can be reduce from $M^2$ to $M$. In this case, core 1 will be the most instruction-saving.

% \subsubsection{The higher order IIR system}

% Let's discuss the higher order IIR system by cascading the proposed three second order IIR cores.
% Take fourth order IIR system as an example.
% Again, if multiple CPUs and threads are used, the cascade of core 1 will be the most efficient. 
% If not, the most instruction-saving combination is core 2 followed by core 3, i.e.,

% \begin{equation*}
%     \label{eq:iir_system_4_core2_core3}
%     \begin{aligned}
%     % \bm{X} = \left[\begin{array}{c|c|c|c}
%     % x[0] & x[M] & \cdots & x[M^2{-}M] \\ 
%     % x[1] & x[M{+}1] & \cdots & x[M^2{-}M{+}1] \\
%     % \vdots & \vdots & \ddots & \vdots \\
%     % x[M{-}1] & x[2M{-}1] &\cdots & x[M^2{-}1] \\
%     % \end{array}\right]
%     & \underrightarrow{\bm{X}_{(1)}} \\
%     & \begin{array}{|c|c|c|c|c|c|c|c}
%         \cline{1-1}
%         \cline{3-3}
%         \cline{5-5}
%         \cline{7-7}
%         % \cline{10-10}
%         \text{T} & \underrightarrow{\bm{X}_{(1)}^T} & \text{T FIR} & \underrightarrow{\bm{V}_{(1)}^T} & \text{T ZIC} & \underrightarrow{\bm{W}_{(1)}^T} & \text{T ICC} & \underrightarrow{\bm{Y}_{(1)}^T(\bm{X}_{(2)}^T)} \\
%         \cline{1-1}
%         \cline{3-3}
%         \cline{5-5}
%         \cline{7-7}
%         % \cline{10-10}
%     \end{array} \\
%         & \begin{array}{c|c|c|c|c|c|c|c|c}
%             \cline{2-2}
%             \cline{4-4}
%             \cline{6-6}
%             \cline{8-8}
%             \underrightarrow{} & \text{T FIR} & \underrightarrow{\bm{V}_{(2)}^T} & \text{T ZIC} & \underrightarrow{\bm{W}_{(2)}^T} & \text{T} & \underrightarrow{\bm{W}_{(2)}} & \text{T ICC}  & \underrightarrow{\bm{Y}_{(2)}} \\
%             \cline{2-2}
%             \cline{4-4}
%             \cline{6-6}
%             \cline{8-8}
%     \end{array}
% \end{aligned}
% \end{equation*}
% The cost for the above fourth order IIR is $2M{\log_2}M{+}16M{+}6 = 182$. Thus, we can see that, for a $L(\geq 4)$th order IIR filter, if only a single CPU and thread is used,
% the optimal combination should be $L/2{-}1$ cascaded second order core 2 followed by one core 3 at the end.

% \subsection{Base function optimization}

% \begin{table}[t]
%     \caption{The number of cycles for each SIMD instruction in VCL}  % title of Table
%     \centering % used for centering table
%     \begin{tabular}{c c c c c c} % centered columns (4 columns)
%     \hline\hline %inserts double horizontal lines
%      & FMA & shuffle & broadcast & load & store \\ [1ex] % inserts table
%     \hline % inserts single horizontal line
%     Number of cycles & 4 & 2-3 & 2-3 & 0-1 & 0-1 \\ [0.3ex]
%     Throughput & 0.5 & 1 & 0.5 & 0.25 & 0.25 \\ [1ex]
%     \hline
%     \end{tabular}
%     \label{table:number_of_cycles} % is used to refer this table in the text
% \end{table}

% We should note that the problem that we discussed so far has two assumptions: 1. The number of cycles for broadcast,
% load and store operations are ignored. This will be true if the three operations take less cycles than FMA and shuffle since
% those operations will be inserted between FMAs and executed within the duration of FMAs. 
% 2. One FMA or shuffle instruction only takes one cycle. However, this is not always true.
% The cost of every SIMD instruction depends on what platform and micro-architecture are used to be implemented. For example, 
% in this paper, we implement the SIMD instructions on software via C++ programming by the mean of vector class library (VCL).
% The computer has the Intel Skylake CPU. The number of cycles that each instruction actually takes is listed in Table \ref{table:number_of_cycles},
% where the (reciprocal) throughput means the maximum number of instructions can be executed per clock cycle, e.g., the 0.5 throughput of FMA means
% at most 2 FMAs can be executed in one clock cycle.

% Because the FMA and shuffle instructions need to take more than one cycle in VCL, different computation strategy will affect the efficiency of base functions.
% We propose below three basic methods of improving the efficiency of base functions:

% 1. Interleave dependency. When the function of transposed ZIC in \eqref{eq:ZIC_w_trans} is executed, the inherent dependency should be paid attention to.
% By unrolling the loop and exchanging the order of FMAs for SIMD vector, \eqref{eq:ZIC_w_trans} can be computed more efficiently as

% \begin{equation}
%     \label{eq:interleave_dependency}
%     \begin{aligned}
%         & \bm{W}^T[0] = \bm{V}^T[0]; \\
%         & \bm{W}^T[1] = \text{mul\_add}(\bm{V}^T[0],a_1,\bm{V}^T[1]); \\
%         & \bm{W}^T[2] = \text{mul\_add}(\bm{W}^T[0],a_2,\bm{V}^T[2]); \\
%         & \bm{W}^T[3] = \text{mul\_add}(\bm{W}^T[1],a_2,\bm{V}^T[3]); \\
%         & \bm{W}^T[2] = \text{mul\_add}(\bm{W}^T[1],a_1,\bm{W}^T[2]); \\
%         & \bm{W}^T[3] = \text{mul\_add}(\bm{W}^T[2],a_1,\bm{W}^T[3]); \\
%         & \bm{W}^T[4] = \text{mul\_add}(\bm{W}^T[2],a_2,\bm{V}^T[4]); \\
%         & \quad\quad\quad\quad\quad\quad\quad\quad\quad \vdots
%     \end{aligned}
% \end{equation}
% where mul\_add($\cdot$) is the FMA instruction for SIMD operation in VCL.
% We can see that by exchanging the order of FMAs, each FMA is interleaved by two cycles, i.e.,
% the efficiency is increased roughly by two times compared with computing in order. 

% 2. Parallelize large sum-up. The expensive part of computing transposed ICC is the big matrix multiplication in \eqref{eq:ICC_init_w_trans}.
% The computation is essentially summing up all the FMAs without dependency for each of two blocks. Thus, we can compute and sum up the FMAs in parallel, and then
% add the sub sum-up together. For example, when calculate \eqref{eq:ICC_init_w_trans} in parallel two,

% \begin{equation}
%     \label{eq:parallelized_sum_up}
%     \begin{aligned}
%         & \bm{yt} = \text{mul\_add}(\bm{T}[0],\bm{W}^T[M{-}2][0],\bm{yt}); \\
%         & \bm{ys} = \text{mul\_add}(\bm{T}[M],\bm{W}^T[M{-}1][0],\bm{ys}); \\
%         & \bm{yt} = \text{mul\_add}(\bm{T}[1],\bm{W}^T[M{-}2][1],\bm{yt}); \\
%         & \bm{ys} = \text{mul\_add}(\bm{T}[M{+}1],\bm{W}^T[M{-}1][1],\bm{ys}); \\
%         & \bm{yl} = \text{mul\_add}(\bm{T}[2M],\bm{W}^T[M{-}2][0],\bm{yl}); \\
%         & \bm{yk} = \text{mul\_add}(\bm{T}[3M],\bm{W}^T[M{-}1][0],\bm{yk}); \\
%         & \bm{yl} = \text{mul\_add}(\bm{T}[2M{+}1],\bm{W}^T[M{-}2][1],\bm{yl}); \\
%         & \bm{yk} = \text{mul\_add}(\bm{T}[3M{+}1],\bm{W}^T[M{-}1][1],\bm{yk}); \\
%         & \quad\quad\quad\quad\quad\quad\quad\quad\quad \vdots
%     \end{aligned}
% \end{equation}
% where $\bm{yt}$, $\bm{ys}$ are two SIMD temporary registers for calculating $\bm{Y}^T[M{-}2]$, and
% $\bm{yl}$, $\bm{yk}$ are registers for calculating $\bm{Y}^T[M{-}1]$. It can be seen that by computing the 
% large sum-up in parallel, we increase the number of FMAs computed simultaneously to approach the throughput of FMA.
% Note, we can also exchange the order of computing $\bm{yt}$, $\bm{ys}$ and $\bm{yl}$, $\bm{yk}$ 
% to improve the efficiency as interleaving.

% 3. Mix FIR and ZIC. When construct a high order IIR filter, as we discussed, core 2 based on transposed data will be used 
% for multiple times. Compared with transposed ICC, transposed FIR and ZIC functions are much cheaper, thus may induce a large overhead.
% On the other hand, to mix executing FIR and ZIC, the inherent dependency of ZIC can be further interleaved by the operations in FIR,

% \begin{equation}
%     \label{eq:mix_fir_zic}
%     \begin{aligned}
%         & \bm{xp2} = \text{blend}(\bm{X}^T[M{-}2],x_{-2}); \\
%         & \bm{xp1} = \text{blend}(\bm{X}^T[M{-}1],x_{-1}); \\
%         & \bm{V}^T[0] = \text{mul\_add}(\bm{xp2},b_2,\bm{X}^T[0]); \\
%         & \bm{V}^T[0] = \text{mul\_add}(\bm{xp1},b_1,\bm{V}^T[0]); \\
%         & \bm{W}^T[0] = \bm{V}^T[0]; \\
%         & \bm{V}^T[1] = \text{mul\_add}(\bm{xp1},b_2,\bm{X}^T[1]); \\
%         & \bm{V}^T[1] = \text{mul\_add}(\bm{X}^T[0],b_1,\bm{V}^T[1]); \\
%         & \bm{W}^T[1] = \text{mul\_add}(\bm{V}^T[0],a_1,\bm{V}^T[1]); \\
%         & \quad\quad\quad\quad\quad\quad\quad\quad\quad \vdots
%     \end{aligned}
% \end{equation}
% where blend is the shuffle instruction in VCL. 


\section{Experimental Results}
\label{sec:experiment}


% \section{Conclusion}

% A family of algorithms for joint detection and carrier synchronization
% suitable for LPI/LPD communications was presented.
% These algorithms are designed for low complexity while achieving good
% performance over a wide range of SNR.\@
% We implemented our algorithms on a SDR platform and demonstrated that
% they can sustain several \si{\mega\hertz} sample rate.

\bibliographystyle{ieeetr}
\bibliography{ref.bib}

\end{document}



%%% Local Variables:
%%% mode: latex
%%% TeX-master: t
%%% End:
