\documentclass[10pt,final,conference,comsoc]{IEEEtran}

\usepackage[utf8]{inputenc}
\usepackage[T1]{fontenc} % optional
\usepackage{amsmath,amssymb,amsfonts}
\usepackage[cmintegrals]{newtxmath}
\usepackage{cite}
\usepackage{algorithmic}
\usepackage{graphicx}
\usepackage{textcomp}
\usepackage{xcolor}
\usepackage{bm}
\usepackage{siunitx}
\usepackage{comment}
\usepackage{mathtools}
\usepackage{url}

\newcommand{\LRT}[2]{\mathrel{\mathop\gtrless\limits^{#1}_{#2}}}
\newcommand{\cn}{\mathcal{CN}}
\newcommand{\n}{\mathcal{N}}
\newcommand{\D}{\mathcal{D}}

\def\BibTeX{{\rm B\kern-.05em{\sc i\kern-.025em b}\kern-.08em
    T\kern-.1667em\lower.7ex\hbox{E}\kern-.125emX}}

\graphicspath{{../main_matlab_figures/}{../matlab/}}

\DeclareUnicodeCharacter{0303}{}

\DeclareSIUnit{\sample}{S}
\DeclareSIUnit{\bits}{b}

\begin{document}

%\title{Low Complexity Methods for Joint Detection and Synchronization of TDMA Bursts}
\title{Practical Methods for Joint Time and Carrier Synchronization in
  LPI/LPD Communications}
\author{\IEEEauthorblockN{Haotian Zhai and Bernd-Peter Paris}
\IEEEauthorblockA{\textit{Department of Electrical and Computer Engineering} \\
\textit{George Mason University}\\
Fairfax, VA 22030 \\
\{hzhai,pparis\}@gmu.edu}}

\maketitle

\begin{abstract}
  To operate at low SNR, LPI/LPD communication systems rely on
  coherent processing which requires that the receiver is precisely
  synchronized in time, frequency and phase. 
  To that end, this paper proposes a family of data-aided joint
  frame and carrier synchronization algorithms.
  A sequential detection algorithm based on the generalized likelihood ratio test (GLRT)
  is used to detect an embedded preamble signal in the received data
  stream.
  Novel algorithms for low-complexity, coarse carrier synchronization
  at low SNR provide carrier estimates
  during sequential detection.
  After detection,
  the coarse carrier estimate is refined for use in coherent demodulation.
  The proposed family of algorithms can be scaled to support operation
  over a wide range of SNR, including SNR below \SI{0}{\decibel}.
  Algorithms are validated through simulation.
  The practicality of our approach is demonstrated by real-time operation on
  a standard SDR platform with sample rates approaching
  \SI{10}{\mega\hertz}.
\end{abstract}

\begin{IEEEkeywords}
LPI/LPD, carrier synchronization, frame synchronization, sequential
detection, GLRT, SDR
\end{IEEEkeywords}

\section{Introduction}
\label{sec:introduction}

Recursive filters are used frequently in digital signal processing.
The main advantage of a recursive filter is the low computation efficiency in implementation.
In order to meet the same specification in terms of passband, stopband, ripple, or roll-off,
recursive filter needs a smaller number of taps than non-recursive filter, which implies
a correspondingly fewer number of computations per unit time. 
The computational savings is often of a rather large factor.

However,
the inherent dependency of recursive filter limits its computation speed.
To overcome the difficulty, the normal way is to dealing with the recursive equation by seperating into
particular and homogeneous solutions.
One previous method based on SIMD operation includes block filtering \cite{Sung_86}, 
which basically computes consecutive samples in one SIMD vector by matrix multiplication. 
The number of arithmetic operation per sample is regardless of the block length. Thus, the block filtering method achieves
intra-block parallelism. 

One more efficient algorithm that exploits inter-block parallelism is to solve the particular part of recursive equation via multiple blocks \cite{Jaewoo_09}.
The consecutive samples are partitioned into multiple blocks in the distance of block length.
% , thus, the multi-block filtering method processes a squared matrix of samples in transposed grid. 
Compared with block filtering, the number of arithmetic operation per sample for multi-block filtering is decreased 
by the reciprocal of the block length. 

There are also many algorithms proposed for efficiently implementing the recursive filter, e.g., including
space-time transformation of loop iterations \cite{Schaffer_03}, tensor product representation \cite{Robelly_04} and
poles reconfiguration \cite{Liu_16}.
However, there is a gap that most of the literatures ignore the rounding error issue \cite{Christoph_97} happens at higher order recursive filter when implementing the algorithms in direct form.
Since the poles depend on all the coefficients of recursive equation in direct form, the finite precision of rounded coefficients may cause some poles near the unit circle
moving outside the unit circle. The solution is to breaking up the higher order filter into cascaded sections, thus the poles are only affected by few coefficients to mitigate
the rounding error issue.

To address both accuracy (filter stability) and efficiency, 
this paper proposes an algorithm of forming the recursive filter by cascading second order sections
align with a novel multi-block filtering approach.
Specifically, compared with \cite{Jaewoo_09}, we extend the multi-block filtering method on solving both particular and homogeneous parts. With the idea of recursive doubling \cite{Kogge_73},
the proposed algorithm solves for the homogeneous part very efficiently. More importantly, the algorithm behind this paper is very suitable for cascaded form of recursive filter.




\section{Signal Model}%
\label{sec:model}

% \begin{figure}[t]
%   \centerline{\includegraphics[width=3.4in]{data_structure.png}}
%   \caption{Structure of signal stream at the receiver}
%   \label{fig:data_structure}
%   \end{figure}

In our model, the transmitted bursts of the received stream are separated by an unkown length of idle period and assumed to
include a reference signal (preamble) that is known to the receiver.
% see Figure~\ref{fig:data_structure}.
%Often such a reference sequence is prepended to the payload and is referred to as a preamble.
%The structure of signal stream at the receiver is shown in Figure~\ref{fig:data_structure}. 
The problem addressed in this paper is to estimate accurately the
start times of the preamble and to estimate carrier phase and frequency
offset from the preamble. 
The payload portion of the burst, which contains the user's information, is not further considered. 
% Moreover, to make our algorithm applicable for signal transmission in very high speed,
% the complexity of algorithm is also crucial.

%We now give the signal model for this paper.
The received signal is modeled at base band.
In~\cite{Morelli_Mengali_98}, 
the authors obtain a simplifed signal model from the matched filter
outputs;
however, this assumes inherently that the symbol time is perfectly known.
Also, the matched filter frontend is not optimal in the presence of
frequency offset.
In contrast, our model assumes merely that the received signal is
oversampled: we collect $M$ samples per symbol period $T$, i.e., the
sample period is $T_s=\frac{T}{M}$. 
Received samples $r_n$ are given by
\begin{equation}
    \begin{aligned}
      \label{eq:model}
      r_p = s_{p-\bar{p}}Ae^{j\phi}e^{j2\pi\delta p}+w_{p},
    \end{aligned}
  \end{equation}
where the reference sequence (preamble) $s_n$ is constructed from
$L_0$ known reference symbols $c_i$ using pulse shaping $g(t)$
\begin{equation}
  \label{eq:l_ref_sig_discrete}
  s_n=\sum_{i=0}^{L_0-1} c_i g(nT_s-iT) \quad \text{for}~n=0,\ldots,N-1.
  % \begin{dcases}
  %   \sum_{i=0}^{L_0-1} c_i g(nT_s-iT) & \text{for}~n=0,\ldots,N-1, \\
  %   0 & \text{otherwise}.
  % \end{dcases}
\end{equation}
The preamble consists of $N=M L_0$ samples.  
  
In~\eqref{eq:model}, $\bar{p}$ denotes the start
position\footnote{This implies that delay is quantized to
  $\bar{p}T_s$. We assume that the signal is oversampled sufficiently
  that the quantization error is negligible.} of the received preamble.
% Note, because of the uncertainty of sampling, often the sampler may not sample exactly at the start time of the preamble, which causes
% the integer delay $\bar{p}$ with a fractional delay in the range of $[-\frac{T_s}{2},\frac{T_s}{2})$, where $T_s$ is the sample period as in~\eqref{eq:l_ref_sig_discrete}.
% In this paper, the sampling rate is assumed to be high enough relative to the symbol rate, so that the influence of fractional delay can be ignored.
$A$, $\phi$, $\delta$ are the amplitude, carrier phase and normalized
(by $T_s$)
frequency offset and
%that we want to estimate.
$w_p$ is complex AWGN.\@
Moreover, we will denote by $E_s/N_0$ the ratio of signal energy to noise power spectral density (SNR).
To simplify analysis, we assume a constant and normalized envelope of the samples in the 
preamble, i.e., $A^2|s_n|^2\approx A^2=E_s/M$ for $n=[0,N-1]$.

In this paper, two time axes are relevant. A global time axis, denoted
by $p$, measures the position of samples in the entire received
stream. A local time axis, denoted by $n$, refers to sample indices
within the preamble (see~\eqref{eq:model}
and~\eqref{eq:l_ref_sig_discrete}).
The local time-axis $n$ is also used in each step of the sequential
detector where a window of $N$ samples is processed.

% Finally, note that in~\eqref{eq:l_ref_sig_discrete}, when the discrete time index $n$ of $s_n$ is greater than $N-1$,
% or equivalently, $n>p+N-1$ in~\eqref{eq:model}, it means the received samples $r_n$ at those $n$ are in the payload;
% We assume the data of payload is zero for simplicity (normally it isn't) because it does not affect the time synchronization of the preamble and carrier synchronization from the preamble.


% The rest of paper includes two main sections. The first section (Section 3 and 4) focus on 
% analyzing the signal acquisition chain, which includes the sequential detection process and
% carrier synchronization of the preamble. The above block diagram is shown in Figure 2.
% The simulation section 5 then illustrates the performance of proposed algorithm
% in the first section. The second section (section 6) moves attention on implementing the algorithm on software-defined radio (SDR).
% Some steps (equations) in the first section are computed more efficiently to achieve the best throughput.

\section{Detection and Time Synchronization}%
\label{sec:detection}

\begin{figure}[t]
  \centerline{\includegraphics[width=2.75in]{partial_preamble_detection.png}}
  \caption{The current window with local time-scale $n$ may contain a
    partial preamble. The preamble starts at $\bar{p}$ on global scale.
    The start of the preamble relative to the window is $\Delta$. The cases $\Delta >0$ and $\Delta < 0$
    are illustrated.}
  \label{fig:partial_preamble_detection}
  \end{figure}

% In this section, we analyze the sequential detection problem, where 
% the detection algorithm proceeds sequentially and each step a window of
% $N$ received samples with $N{-}1$ overlapped samples from the previous window is considered.
We start by looking at the two hypotheses for the sequential detection task:
Let $H_0$ be the null hypothesis that the received signal is the
channel noise or only contains a portion of the preamble against the
alternative $H_1$ that it contains the entire preamble.  
% The detection algorithm proceeds sequentially and each step a window of
% length $N$ received samples with $N-1$ overlapped samples from the previous window is considered.
Define $\Delta$ to be the distance between the current start position of the window
of $N$ received samples and the start position of the preamble $\bar{p}$.
Fig.~\ref{fig:partial_preamble_detection} illustrates two distinct cases when a portion of the 
preamble is in the window.
%with global delay index $p$ and local sample index $n$.
In terms of the local scale $n=0,1,\ldots,N-1$, the two hypotheses are given by the window of samples $r_n$
\begin{equation}
  \label{eq:two hypotheses}
  \begin{aligned}
  &H_0{:}~r_n{=}
  \begin{dcases}
      s_{n-\Delta}\xi e^{j2\pi\delta (n-\Delta)}{+}w_n & \max(0{,}\Delta){\leq} n{<} \min(N{,}N{+}\Delta) \\
      % s_{n-\Delta}\xi e^{j2\pi\delta (n-\Delta)}{+}w_n & n{\in}[\max(0,\Delta), \min(N,N{+}\Delta)) \\
      w_n & \text{otherwise},
  \end{dcases} \\
  &H_1{:}~r_n{=}s_n\xi e^{j2\pi\delta n}+w_n.
  \end{aligned}
\end{equation}
where $\xi=Ae^{j\phi}$ denotes the phasor in~\eqref{eq:model}.
Furthermore, $\Delta \neq 0$ is the premise under hypothesis $H_0$
while $\Delta=0$ under $H_1$.  
% Figure 3 demonstrates the case when $H_0$ represents a portion of the preamble compared with
% $H_1$. 

We focus on discussing when the start of the preamble 
occurs in the window, i.e., $\Delta\in(0,N{-}1]$, which corresponds to
the top case in Fig.~\ref{fig:partial_preamble_detection}.
The bottom case when 
$\Delta\in[-N{+}1,0)$ 
is symmetric to the top case.
Based on~\eqref{eq:two hypotheses}, we build conditional likelihood ratio test (CLRT)
between $H_0$ and $H_1$ by conditioning on the distance $\Delta$, 
the phasor $\xi$ and the frequency offset $\delta$. The conditional log likelihood ratio reduces to
% is given by
% \begin{equation}
%     \label{eq:likelihood ratio}
%     \begin{aligned}
%     \Lambda(R|\Delta,\xi,\delta)&=\frac{f_{R|H_1,\xi,\delta}(r|H_1,\xi,\delta)}{f_{R|H_0,\Delta,\xi,\delta}(r|H_1,\Delta,\xi,\delta)} \\
%     &=\frac{\displaystyle \prod_{n=0}^{N-1}\frac{1}{\sqrt{\pi N_0}}e^{-\frac{|r_n-s_n\xi e^{j2\pi\delta n}|^2}{N_0}}}{\displaystyle \frac{1}{(\pi N_0)^{N/2}}\prod_{n=\Delta}^{N-1}e^{-\frac{|r_n-s_{n-\Delta}\xi e^{j2\pi\delta (n-\Delta)}|^2}{N_0}}\prod_{n=0}^{\Delta-1}e^{-\frac{|r_n|^2}{N_0}}} \\
%     % &=\frac{\prod_{n=0}^{N-1}\frac{1}{\sqrt{\pi N_0}}e^{-\frac{|r_n-s_n\xi e^{j2\pi\delta n}|^2}{N_0}}}{\prod_{n=\Delta}^{N-1}\frac{1}{\sqrt{\pi N_0}}e^{-\frac{|r_n-s_{n-\Delta}\xi e^{j2\pi\delta (n-\Delta)}|^2}{N_0}}\prod_{n=0}^{\Delta-1}\frac{1}{\sqrt{\pi N_0}}e^{-\frac{|r_n|^2}{N_0}}} \\
%     % &\times \frac{1}{\prod_{n=0}^{\Delta-1}\frac{1}{\sqrt{\pi N_0}}e^{-\frac{|r_n|^2}{N_0}}}
%     & \LRT{H_1}{H_0} \eta.
%     \end{aligned}
%   \end{equation}
  % Canceling the common parts and taking the logarithm,~\eqref{eq:likelihood ratio} is reduced to
\begin{equation}
    \label{eq:log likelihood}
    \begin{aligned}
    \Re\Bigg\{\sum_{n=0}^{N-1}r_ns_n^*\xi^*e^{-j2\pi\delta n}-\sum_{n=\Delta}^{N-1}r_n&s_{n-\Delta}^*\xi^*e^{-j2\pi\delta(n-\Delta)}\Bigg\} \LRT{H_1}{H_0} \\
    &\frac{N_0}{2}\ln\eta+\frac{A^2}{2}\sum_{n=N-\Delta}^{N-1}|s_n|^2.
    \end{aligned}
\end{equation}

On the left hand side, the two summations are the matched filters for hypothesis $H_1$
and $H_0$, respectively. 
It can be shown that the principal value of the left hand side of~\eqref{eq:log likelihood}
reduces to the difference between the energy of the preamble and a partial
autocorrelation function (ACF) of the preamble at lag $\Delta$ under
the two hypotheses.
Moreover, we notice the second summation on the left hand side is not computable
due to the unknown information of $\Delta$ while the first summation is.
% On the other hand,
% the first summation reflects the energy of the preamble
% under $H_1$ and the partial ACF of the preamble at lag $\Delta$ under $H_0$.
To render the effect of the partial ACF negligible, a preamble
with very good autocorrelation properties is critical.

Then, a practical sequential detector reduces to the
cross-correlation between the received signal
and the preamble corrected by the frequency and phasor estimates with
proper scaling.
Specifically, for each window starting at global sample index $p$
\begin{equation}
  \label{eq:generalized_corr}
  \rho(p)=
  \frac{\Re\{\langle
    \bm{r}_{p},\hat{\bm{s}}_{p}\rangle\}}
  {||\bm{r}_{p}||\cdot||\hat{\bm{s}}_{p}||} \LRT{H_1}{H_0} \gamma
\end{equation}
where $\bm{r}_{p}{=}[r_{p},r_{p+1},\ldots,r_{p+N-1}]$ denotes the window of received signal
starting at position $p$. $\hat{\bm{s}}_{p}$ denotes the carrier-corrected preamble,
where each element is $\hat{s}_{n}=s_n\hat{\xi}_{p}e^{j2\pi\hat{\delta}_{p}n}$
for $n=0,1,\ldots,N{-}1$, and $\hat{\xi}_{p}$, $\hat{\delta}_{p}$ are the carrier estimates at
position $p$.
%$||\cdot||$ is the Euclidean norm of the signal sequence, and
The normalized detection threshold $\gamma$ lies in the range of $[0,1]$. 
From~\eqref{eq:generalized_corr}, we see
a realistic generalized likelihood ratio test (GLRT) replaces the CLRT
by first performing  carrier estimation of the signal in window at
each position $p$
and then applying corrections based on these estimates in the LRT.\@
It should be emphasized that our detector is intended to work at high sample rates.
While~\eqref{eq:generalized_corr} is straightforward and of complexity 
$O(N)$, the complexity of estimating $\hat{\xi}$,$\hat{\delta}$
is critical for practical implementation.  
A pair of low-complexity frequency and phasor estimates will be given in the next section.


% On the left hand side, the two summations are the matched filters for hypothesis $H_1$
% and $H_0$, respectively. It is easy to derive that the principal value of the log-CLRT
% reduces to the difference between the energy of the (partial) preamble and a partial
% autocorrelation function (ACF) of the preamble at lag $\Delta$ under two hypotheses.

% It can be also seen that the second summation of~\eqref{eq:log likelihood} is not computable
% due to the unknown information of the distance $\Delta$.

% To explain this, we plug $r_n=s_n\xi e^{j2\pi\delta n}+w_n$ under hypothesis $H_1$ into the left hand side of~\eqref{eq:log likelihood}.
% In the real operator, it yields

% \begin{equation}
%     \label{eq:log likelihood under H_1}
%     \begin{aligned}
%     % &\sigma(\Delta) \\    
%     % &=A^2\sum_{n=0}^{N-1}|s_n|^2-A^2e^{j2\pi\delta\Delta}\sum_{n=\Delta}^{N-1}s_ns_{n-\Delta}^*+\text{a zero-mean AWGN}
%     A^2\sum_{n=0}^{N-1}|s_n|^2-A^2e^{j2\pi\delta\Delta}\sum_{n=\Delta}^{N-1}s_ns_{n-\Delta}^*+\text{zero-mean noise},
%     \end{aligned}
% \end{equation}
% where $A^2\sum_{n=0}^{N-1}|s_n|^2$ denotes the energy of the preamble and
% $\sum_{n=\Delta}^{N-1}s_ns_{n-\Delta}^*$ is the partial autocorrelation function (ACF) of the
% preamble at lag $\Delta$. Similarly, it can be derived the left hand side of~\eqref{eq:log likelihood}
% under hypothesis $H_0$ in the real operator is 
% % $-A^2\sum_{n=\Delta}^{N-1}|s_{n-\Delta}|^2+A^2e^{-j2\pi\delta\Delta}\sum_{n=\Delta}^{N-1}s_{n-\Delta}s_n^*$

% \begin{equation}
%     \label{eq:log likelihood under H_0}
%     \begin{aligned}
%     % &\sigma(\Delta) \\    
%     % &=A^2\sum_{n=0}^{N-1}|s_n|^2-A^2e^{j2\pi\delta\Delta}\sum_{n=\Delta}^{N-1}s_ns_{n-\Delta}^*+\text{a zero-mean AWGN}
%     -A^2\sum_{n=\Delta}^{N-1}|s_{n-\Delta}|^2+A^2e^{-j2\pi\delta\Delta}\sum_{n=\Delta}^{N-1}s_{n-\Delta}s_n^*+\text{zero-mean noise}.
%     \end{aligned}
% \end{equation}
% Thus, based on~\eqref{eq:log likelihood under H_1} and~\eqref{eq:log likelihood under H_0},
% we can conclude that the principle value of the log-CLRT reduces to the difference between
% the energy of the (partial) preamble and the ACF of the preamble sequence at lag $\Delta$. 
% By using the symbol sequence $\{c_i\}$ of $s_n$ with a good autocorrelation property, e.g., Gold sequence, the
% effect of the ACF can be kindly mitigated; However, because of pulse shaping,
% the autocorrelation property of the preamble is decreased so that 
% the effect of the ACF cannot be ignored. 

% In practice, the distance $\Delta$ is an unknown information to the receiver,
% which means the second summation of~\eqref{eq:log likelihood} is not computable.
% However, the first summation is computable and it reflects the energy of the preamble under
% $H_1$ and the partial ACF under $H_0$.
% Thus, a practical sequential detector is built just based on the classical 
% cross-correlation between the received signal
% and the preamble corrected by the frequency and phasor estimates with some proper scaling. Specifically,

% \begin{equation}
%     \label{eq:generalized_corr}
%     \rho(\tilde{p})=
%     \frac{\Re\{\langle
%       \bm{r}_{\tilde{p}},\hat{\bm{s}}_{\tilde{p}}\rangle\}}
%     {||\bm{r}_{\tilde{p}}||\cdot||\hat{\bm{s}}_{\tilde{p}}||} \LRT{H_1}{H_0} \gamma
%   \end{equation}
% where $\tilde{p}{=}p{+}\Delta$ denotes the general start position of received signal. 
% $\bm{r}_{\tilde{p}}{=}[r_{\tilde{p}},r_{\tilde{p}+1},\ldots,r_{\tilde{p}+N-1}]$ represents the received signal sequence at
% the position $\tilde{p}$, and $\hat{\bm{s}}_{\tilde{p}}$ represents the carrier-estimates corrected preamble sequence, where each element is 
% $\hat{s}_{n}=s_n\hat{\xi}_{\tilde{p}}e^{j2\pi\hat{\delta}_{\tilde{p}}n}~\text{for}~n=0,1,\ldots,N-1$, and $\hat{\xi}_{\tilde{p}}$, $\hat{\delta}_{\tilde{p}}$
% are the phasor and frequency estimates at position $\tilde{p}$. Moreover, $||\cdot||$ is the Euclidean norm of the signal sequence,
% and $\gamma$ is the normalized detection threshold which lies on the range of $[0,1]$. 
% From~\eqref{eq:generalized_corr}, it is obvious that to implement the sequential detector, we need the information
% of the frequency and phasor estimates at that position. Thus, 

% a realistic generalized likelihood ratio test (GLRT) replaces the CLRT of~\eqref{eq:likelihood ratio}
% by first doing carrier synchronization at each position of detection then plugging the estimates into the LRT.
% Furthermore, it should be emphasized that our sequential detector requires to work at high sample rate; Although~\eqref{eq:generalized_corr}
% is already the simpliest form, the complexity of the two estimates $\hat{\xi}$,$\hat{\delta}$ is also crucial. 
% A pair of low-complexity frequency and phasor estimates will be given in the next section.







\section{Frequency and Phase Estimation}%
\label{sec:freq_est}   

In this section, estimation is performed using a window of $N$ samples $r_n$, $0\leq n<N$, starting
at the position $\bar{p}$ of the preamble.
For estimating the frequency offset $\delta$ and phasor
$\xi=Ae^{j\phi}$, the maximum likelihood estimation (MLE) of the
parameters in~\eqref{eq:model} is given by 
\begin{equation}
  \label{eq:ML_f_xi}
  \hat{\delta},\hat{\xi}=\min_{\delta,\xi=Ae^{j\phi}}\sum_{n=0}^{N-1}|r_n-s_n\xi e^{j2\pi\delta n}|^{2}.
\end{equation}

A closed form for $\hat{\xi}$ is readily derived by taking the
Wirtinger derivative with $\xi$ and setting it equal to zero, 
\begin{equation}
  \label{eq:opt_xi}
  \hat{\xi}=\frac{\sum_{n=0}^{N-1}{r_{n}s_n^{*}e^{-j2\pi\hat{\delta} n}}}{\sum_{n=0}^{N-1}|s_{n}|^2},
\end{equation}
so that $\hat{\xi}$ relies on the frequency estimate $\hat{\delta}$; $\hat{\phi}=\arg\{\hat{\xi}\}$. 
% Moreover, by plugging~\eqref{eq:opt_xi} into~\eqref{eq:generalized_corr}, the GLRT based detector can be computed more
% efficiently,

% \begin{equation}
%     \label{eq:simplified_GLRT_detector}
%     \rho(p{+}\Delta) \approx
%     \frac{|\tilde{\xi}_{p+\Delta}|}
%     {||\bm{r}_{p+\Delta}||\cdot||\bm{s}||} \LRT{H_1}{H_0} \gamma
%   \end{equation}
% where $\tilde{\xi}_{p+\Delta}=\sum_{n=0}^{N-1}{r_{n+p+\Delta}s_n^{*}e^{-j2\pi\hat{\delta}_{p+\Delta} n}}$ represents 
% the cross-correlation (the numerator) of the phasor estimate $\hat{\xi}$ in~\eqref{eq:opt_xi} at position $p{+}\Delta$, and $||\bm{s}||$ is the Euclidean norm of the preamble.

A necessary condition for the frequency offset estimate $\hat{\delta}$
is obtained similarly by taking the derivative of~\eqref{eq:ML_f_xi} with respect to
$\delta$, inserting~\eqref{eq:opt_xi}, and setting the result to
zero.
Skipping all intermediate derivation steps for brevity's sake, this yields
\begin{equation}
  \label{eq:necessary condition for delta}
  J(\hat{\delta}) = \Im\bigg\{\sum_{k=1}^{N-1}{\sum_{m=k}^{N-1}{kr_{m-k}r_m^{*}s_{m-k}^{*}s_m}e^{j2\pi\hat{\delta}k}}\bigg\}=0.
\end{equation}
There are a number of local minima of~\eqref{eq:ML_f_xi} also satisfying
the necessary condition for $\hat{\delta}$ in~\eqref{eq:necessary condition for delta}
in addition to the absolute minimum (the exact solution of MLE).
In~\cite{Luise_Reggiannini_95} and~\cite{Fitz_94}, the ``false minima'' are avoided
by appropriately restricting the operating range of the estimator.
Specifically,
instead of calculating the sample autocorrelation functions for all
lags $k\in[1,N-1]$, 
they truncate~\eqref{eq:necessary condition for delta} by only considering the lags $k{\in}[1,E]$,
where $E \ll N{-}1$.
This is to avoid 
including the inner summations in~\eqref{eq:necessary condition for delta} with few
terms and correspondingly high variance.

Moreover, the estimator $\hat{\delta}$ in~\eqref{eq:necessary condition for delta} has no closed-form solution.
In~\cite{Luise_Reggiannini_95}, the necessary condition is
approximated by replacing the exponential with its Taylor series
expansion. 
In~\cite{Fitz_94}, an approximate solution is obtained via Euler's identity for large $N$.
The estimators (L\&R and Fitz) in both~\cite{Luise_Reggiannini_95} and~\cite{Fitz_94}  have computational complexity $O(N^2)$ 
reflecting the double summation.
In~\cite{kay_89}, the Kay estimator reduces the complexity from $O(N^2)$
to $O(N)$ by only
computing~\eqref{eq:necessary condition for delta} at lag $k=1$.
However, it suffers from poor accuracy even at moderate SNRs.

In this paper, we propose a family of alternative solutions to~\eqref{eq:necessary condition for delta}.
A coarse solution with $O(N)$ complexity is used for operating at high sample rate during the sequential detection.
It prioritizes low complexity at the expense of some loss of
accuracy.
A second more accurate solution is used to improve
the estimation accuracy at moderate complexity to enable coherent demodulation once the preamble has been detected. 

\subsection{Coarse Solution: Single-Lag Estimator with Partial
  Coherent Integration}

The first estimator is rooted in the insight that at high SNR,
every lag $k$ in~\eqref{eq:necessary condition for delta}
can be used to approximate the frequency estimate $\delta$.
By setting $r_m\approx s_m\xi e^{j2\pi\delta m}$,~\eqref{eq:necessary condition for delta} expands to
\begin{equation}
  \label{eq:delta_extens_no_noise}
  \Im\bigg\{A^2\sum_{k=1}^{N-1}\sum_{m=k}^{N-1}k|s_{m-k}|^2|s_m|^2e^{j2\pi (\hat{\delta}-\delta)k}\bigg\}=0.
  \end{equation}
The inner summation in~\eqref{eq:delta_extens_no_noise} is purely real for every lag $k$ if $\hat{\delta}=\delta$.
This suggests that an unbiased estimate of $\delta$ can be obtained by
using only a single lag $k$.
This approach yields
a closed-form solution for $\hat{\delta}$, which is given by
\begin{equation}
  \label{eq:single_lag_estimator_wout_partial_corr}
  \hat{\delta}_{SL}(k)=-\frac{\arg\big\{\sum_{m=k}^{N-1}r_{m-k}r_m^*s_{m-k}^*s_m\big\}}{2\pi k}.
\end{equation}

However, the above single-lag (SL) estimator has insufficient
accuracy at low SNRs as we will show below.
To extend the range of operation to low SNR,
a generalized single-lag estimator that includes partial coherent integration
prior to estimation is given by
\begin{equation}
  \label{eq:single_lag_estimator_w_partial_corr}
  \hat{\delta}_{SL}^{(v)}(k_v)=-\frac{\arg\big\{\sum_{l=k_v}^{N/v-1}\digamma_l^{*(v)}\digamma_{l-k_v}^{(v)}\big\}}{2\pi k_vv},
\end{equation}
where $\digamma_{l}^{(v)}$ denotes the result of partial coherent integration over block $l$
of length~$v$
\begin{equation}
  \label{eq:coherent_integrator}
  \digamma_l^{(v)}=\sum_{n=lv}^{(l+1)v-1}r_ns_n^*, \quad \text{for}~l=0,1,\ldots,N/v{-}1.
\end{equation}

Normally, $v \leq \frac{N}{2}$ is set to be a factor of $N$ to include
all $N$ samples in the window.
In~\eqref{eq:single_lag_estimator_w_partial_corr}, $k_v=\lfloor k/v \rfloor$ denotes
the distance between blocks that are used for measuring phase differences.
It can be seen that 
\eqref{eq:single_lag_estimator_wout_partial_corr} is a special case of~\eqref{eq:single_lag_estimator_w_partial_corr}
when $v=1$, i.e,  no partial integrator is used.

\subsubsection{Performance of single-lag estimator}

In this section we derive expressions for the mean and variance of our
estimators and assess the benefits of partial coherent integration.
While our analysis is specific to the
estimator~\eqref{eq:single_lag_estimator_w_partial_corr}, it
applies equally to any estimator based on~\eqref{eq:necessary
  condition for delta} involving products of received samples.
We first look at
the probability density function (pdf) of the partial coherent integration
$\digamma_l^{(v)}$.
By setting $r_n=s_n\xi e^{j2\pi \delta n}+w_n$ in~\eqref{eq:coherent_integrator},
$\digamma_{l}^{(v)}$ yields a complex Gaussian random variable with pdf
\begin{equation}
  \label{eq:pdf_co_integrator}
  \begin{aligned}
    &\digamma_l^{(v)} \sim \cn\bigg(\xi\sum_{n=lv}^{(l+1)v-1}e^{j2\pi \delta n},\frac{N_0}{2}v\bigg).
  \end{aligned}
\end{equation}

Note, $\digamma_l^{*(v)}$ and $\digamma_{l-k_v}^{(v)}$ are
uncorrelated as they are computed from non-overlapping blocks. 
Based on~\eqref{eq:pdf_co_integrator}, 
the product $C_{\digamma}(v,l)=\digamma_l^*\digamma_{l-k_v}$
has a mixed distribution of a complex Gaussian and a Bessel function
of the second kind; 
the latter stems from the products of noise terms; it cannot be
neglected when SNR is less than \SI{0}{\decibel}. 

The mean $\mu_{C_{\digamma}}$ and variance $\sigma^2_{C_{\digamma}}$ of $C_{\digamma}(v,l)$ are given by, respectively 
\begin{equation}
  \begin{aligned}
  \label{eq:mean_var_product_coherent_int}
  \mu_{C_{\digamma}}&=\frac{E_s}{M}\sum_{n=lv}^{(l+1)v-1}e^{-j2\pi \delta n}\bigg(\sum_{m=(l-k_v)v}^{(l-k_v+1)v-1}e^{j2\pi \delta m}\bigg) \\
  &=E_s/M\cdot e^{-j2\pi \delta k_vv}\D^2(v,\delta), \\
  \sigma^2_{C_{\digamma}}&={\underbrace{N_0^2v^2/4}_{\text{from Bessel}}}+{\underbrace{N_0E_sv/M\cdot\D^2(v,\delta)}_{\text{from Complex Gaussian}}},
  \end{aligned}
\end{equation}
where $\D(v,\delta) \triangleq \frac{\sin(\pi \delta v)}{\sin(\pi
  \delta)}$ is the Dirichlet function of $\delta$, which approaches 
the maximum value $v$ at $\delta=0$ and has its first two zeros at $\delta=\pm 1/v$.
Note, both $\mu_{C_{\digamma}}$ and $\sigma^2_{C_{\digamma}}$ are independent of the partial integration block~$l$.
Thus, the mean and variance of  
$\sum C_{\digamma}$ in the argument operator of~\eqref{eq:single_lag_estimator_w_partial_corr} 
are $\mu_{\sum C_{\digamma}}=\sum\mu_{C_{\digamma}}$ and 
$\sigma^2_{\sum C_{\digamma}}=\sum\sigma^2_{C_{\digamma}}$, respectively.

Recall from~\eqref{eq:single_lag_estimator_w_partial_corr},
the distribution of the SL estimator depends on
$\arg\{\sum C_{\digamma} \}$.
It can be shown that the full pdf of $\arg\{\zeta\}$, when $\zeta$ is
complex Gaussian distributed, is well approximated for moderate SNR,
as Gaussian. 
Specifically,
\begin{equation}
  \label{eq:pdf_arg_comp_Gaus}
  \arg\{\zeta\} \sim \n\big(\angle \mu_{\zeta},\sigma^2_{\zeta}/|\mu_{\zeta}|^2\big).
\end{equation}
% The derivation of~\eqref{eq:pdf_arg_comp_Gaus} is omitted due to space constraint.

Based on~\eqref{eq:pdf_arg_comp_Gaus} and with
$\zeta = \sum C_{\digamma}$,
the SL estimator is optimized by minimizing the
variance $\frac{\sigma^2_{\sum C_{\digamma}}}{|\mu_{\sum C_{\digamma}}|^2}$.
Equivalently, we can maximize the ``output'' SNR, 
\begin{equation}
  \begin{aligned}
    \label{eq:SNR_out}
    \text{SNR}_{\sum C_{\digamma}}^{(v,\delta)}=\frac{|\mu_{\sum C_{\digamma}}|^2}{\sigma^2_{\sum C_{\digamma}}} 
    % &=\frac{\frac{Es^4}{M^4A^4}\Big(\frac{\sin(\pi \delta v)}{\sin(\pi \delta)}\Big)^4(N/v-k_v)^2}
    % {\Big[v^2\frac{N_0^2E_s^2}{4M^2A^4}+2v\frac{N_0E_s^3}{2M^3A^4}\Big(\frac{\sin(\pi \delta v)}{\sin(\pi \delta)}\Big)^2\Big](N/v{-}k_v)} \\
    =\frac{(N/v-k_v)\cdot\D^4(v,\delta)}
    {v^2/\text{SNR}_{\text{in}}+2v\cdot\D^2(v,\delta)}\cdot\text{SNR}_{\text{in}}. \\
  \end{aligned}
\end{equation}
where $\text{SNR}_{\text{in}}=2E_s/(MN_0)$.
Therefore, the block length $\nu$ should satisfy $2\nu \gg
\frac{1}{\text{SNR}_{\text{in}}}$ and $\nu \ll \frac{1}{\delta}$.

From~\eqref{eq:SNR_out}, we 
see that  at low SNR $\text{SNR}_{\sum C_{\digamma}}^{(v,\delta)}$ is degraded by
the variance of the second kind Bessel random variable (the $v^2$ term
in the denominator).
For small frequency offsets  $|\delta|v \approx 0$,
we can compare the performance
with and without partial integration via the ratio
\begin{equation}
  \begin{aligned}
    \label{eq:relative_processing_gain}
    \frac{\text{SNR}_{\sum C_{\digamma}}^{(v,0)}}{\text{SNR}_{\sum C_{\digamma}}^{(1,0)}}
    \approx\frac{v+2v\cdot\text{SNR}_{\text{in}}}{1+2v\cdot\text{SNR}_{\text{in}}}.
  \end{aligned}
\end{equation}

Some observations can be obtained from~\eqref{eq:relative_processing_gain}.
First, we see the maximum relative performance approaches $v$ as
$\text{SNR}_{\text{in}}$ approaches 0.
This underscores that partial integration is critical at low SNR.
Moreover, when the input SNR is fixed and small, the relative performance
increases with $v$.
On the other hand, at high input SNRs the performance ratio approaches~$1$.
Then, the effect of Dirichlet function in~\eqref{eq:SNR_out}
becomes more important and
$v=1$ should be chosen.

Based on~\eqref{eq:single_lag_estimator_w_partial_corr},~\eqref{eq:mean_var_product_coherent_int}
and~\eqref{eq:pdf_arg_comp_Gaus},
the distribution of $\hat{\delta}_{SL}^{(v)}(k_v)$ at moderate or high
(output) SNRs is given by
\begin{equation}
  \label{eq:lower_bound_single_lag_high_snr}
  \hat{\delta}_{SL}^{(v)}(k_v) \sim \n \Bigg(\delta,\frac{(Mv+4\D^2(v,\delta)\cdot E_s/N_0)\cdot Mv}{16\pi^2k^2(N/v{-}k_v)(E_s/N_0)^2\D^4(v,\delta)}\Bigg).
\end{equation}
Thus, $\hat{\delta}_{SL}^{(v)}(k_v)$ is unbiased. 
% At high SNR, a lower bound for the variance of single-lag estimator is 
% the variance of~\eqref{eq:lower_bound_single_lag_high_snr} when $v=1$.
% On the other hand, an approximate lower bound at low SNRs can be obtained by the variance of~\eqref{eq:lower_bound_single_lag_high_snr}
% at maximum $v=N/2$.
Moreover, the variance in~\eqref{eq:lower_bound_single_lag_high_snr} also depends on
the value of $k_v$.
The best choice for $k_v$ is to choose 
$k_v=\lfloor\frac{2N}{3v}\rfloor$ to minimize the variance.

\subsubsection{Estimation range}
The SL estimator may suffer an effect
akin to aliasing when  $2\pi M|\delta|k_vv{>}\pi$.
Thus, a safe estimation range for the SL estimator with optimal
$k_v=\lfloor\frac{2N}{3v}\rfloor$
to avoid modulo-$2\pi$ ambiguity
is $\delta$ within $\pm 3/(4MN)$. 
Compared with the traditional autocorrelation-based estimator,
e.g.,~\cite{Luise_Reggiannini_95} or~\cite{Fitz_94},
the SL estimator has 
$3/8$ estimation range of~\cite{Luise_Reggiannini_95} and $3/4$ estimation range of~\cite{Fitz_94}.

\subsubsection{Computational complexity}

We have discussed the accuracy of the single-lag estimator.
It is also necessary to address the complexity as the SL estimator
is intended for real-time use with high sample rates.
The computational complexity of the single-lag estimator
can be readily assessed
from~\eqref{eq:single_lag_estimator_w_partial_corr},~\eqref{eq:coherent_integrator}
and~\eqref{eq:single_lag_estimator_wout_partial_corr}.

Specifically, we compare the complexity of SL estimators in sequential detection with and without partial correlating.
Without partial correlating, from~\eqref{eq:single_lag_estimator_wout_partial_corr},
$s_{m-k}^*s_m$ can be precomputed and stored.
Moreover, due to the characteristic of the sequential detection
process,
the products of received samples, $r_{m-k}r_m^*$,
can be stored so that only one new product needs to be computed per
sequential detection step (i.e., per sample period).

The  computational complexity of the two single-lag estimators are given
in Table~\ref{table:computational complexity comparison} for optimal $k_v \approx 2N/(3v)$. 
We see that $\hat{\delta}_{SL}^{(1)}(k)$ has approximately two times
fewer complex products and additions than
$\hat{\delta}_{SL}^{(v>1)}(k_v)$. 
Furthermore, note the complexity of the Kay estimator~\cite{kay_89},
also with a single lag,
is approximately $3N/4$ complex products and additions. 
% which is slightly larger than $\hat{\delta}_{SL}^{(1)}(k)$.
Recall that estimators~\cite{Fitz_94,Luise_Reggiannini_95} are $O(N^2)$.

\begin{table}[t]
  \caption{Complexity of single-lag estimators with and without partial correlating in sequential detection with optimal $k_v\approx2N/(3v)$}  % title of Table
  \centering 
  \begin{tabular}{c c c} 
  \hline\hline 
   & Complex products & Complex additions \\ [0.5ex] 
  \hline 
  $\hat{\delta}_{SL}^{(1)}(k)$  & $N/3+1$ & $N/3$ \\ 
  $\hat{\delta}_{SL}^{(v)}(k_v)$ & $(2+1/v) N/3$ & $(2-1/v)N/3-1$ \\ [1ex]
  \hline
  \end{tabular}
  \label{table:computational complexity comparison}
\end{table}

\subsection{Fine Solution: Newton-Method based Estimator}

The SL estimator emphasizes low-complexity  and is intended
to provide merely sufficiently good carrier synchronization to enable coherent detection.
Once the signal has been acquired, the SL estimator can be improved by 
investing additional computations. Since detection events are rare, the computational
complexity is of little concern.

The principle of Newton-method based estimator is to use the
single-lag estimator as the starting point for a Newton-type iteration  
aimed at finding a better solution to the necessary condition~\eqref{eq:necessary condition for delta}. 
In principle, multiple iterations are possible to produce successively better approximations to the root of
$J'(\cdot)$ in~\eqref{eq:necessary condition for delta}.
Specifically, the iterations are given by
\begin{equation}
  \label{eq:iter_NM_est}
  \hat{\delta}_{NM}^{(i+1)}=\hat{\delta}_{NM}^{(i)}-
  \frac{J(\hat{\delta}_{NM}^{(i)})}{J^\prime(\hat{\delta}_{NM}^{(i)})}
\end{equation}
where $\hat{\delta}_{NM}^{(0)}=\hat{\delta}^{(v)}_{SL}(k_v)$ is the
starting point of the iteration and
$J^\prime(\cdot)$ denotes the derivative of $J$ with respect to $\hat{\delta}$. 
Our simulations indicate that  a single iteration is usually sufficient to achieve very good accuracy.

% From~\eqref{eq:iter_NM_est} and the previous discussion, 
% We conclude with a reminder of the importance of accuracy of the SL estimator at low SNRs: 
% with merely sufficiently good accuracy, the SL estimator not only increases the probability of detection by better
% fitting the preamble and received signal as in sequential detector~\eqref{eq:generalized_corr}, but 
% it provides a reasonable starting point for
% getting the more accurate NM estimator. 
% In simluations,
% we will also show the case when the NM estimator has a worse accuracy than the SL estimator if the latter does not provide enough accuracy. 

% ambiguity problem for sl.

\section{Simulation Results}%
\label{sec:simulations}

\begin{figure}[t]
    \centerline{\includegraphics[width=3.15in]{accuracy_NM_SL.png}}
    \caption{Accuracy of the NM and the SL estimators ($L_0=32$, $M=2$)}
    \label{fig:accuracy_NM_SL}
    \end{figure}

In the simulation section, we reverse the order of discussion by first showing 
the accuracy of estimators in carrier synchronization and then showing some results of sequential detection since
the GLRT based detector in~\eqref{eq:generalized_corr} relies on the accuracy of 
the SL estimator.
The symbol sequence of the preamble is chosen as a Gold sequence 
with good autocorrelation properties and
modulated by a QPSK alphabet.
The pulse is chosen as a
Square-Root Raised Cosine (SRRC) pulse with roll-off equal to 0.5.
The normalized frequency offset $\delta$ is intentionally set to be in
the safe estimation range for all estimators for simulation purposes. 

\subsection{Simulation Results for Estimation}%

% \begin{figure}[t]
%     \centerline{\includegraphics[width=3.4in]{generalized_correlation_p_plus_delta.png}}
%     \caption{Performance of sequential detector when the preamble is pulse shaped}
%     \label{fig:Generalized correlation}
%     \end{figure}

Fig.~\ref{fig:accuracy_NM_SL} illustrates the accuracy of the single-lag (SL) estimator and the NM estimator.
Comparing the two curves for the SL estimators with $v=1$ and $v=32$, 
we see the length-$32$ partial integration
improves the accuracy by providing an approximate
\SI{4}{\dB}~performance gain at negative SNRs
(near $\text{SNR}=\SI{-5}{\dB}$). This is consistent
with~\eqref{eq:relative_processing_gain}.

The SL estimator with $v=1$ is slightly more accurate
at \SI{10}{\dB}~SNR than the one with $v=32$.
The gap is due to the Dirichlet function.
% for $\delta \neq 0$.
%For the same reason,
Also, when $v>1$
the accuracy of the estimator improves at all SNRs as $|\delta|$ decreases.
% When  $\delta$ is very small, the accuracy of the SL estimator with $v=32$ has better accuracy at all SNRs.

The SL estimators do not approach the Cramer-Rao Vector Bound (CRVB)~\cite{Gini_98} while the NM
estimators do at moderate SNR. 
% We also see that the
The NM estimator achieves improved accuracy at lower SNR
when the SL estimator with $v=32$ is used as the starting point. 
In contrast, at all negative SNRs the NM estimator based on SL estimation with $v=1$ has
worse accuracy than SL estimation alone.
This is because the accuracy of the SL estimator is not sufficient to
provide a consistently good starting point and the Newton iteration converges occasionally to 
other local minima away from the true frequency offset.

Fig.~\ref{fig:accuracy_NM_SL_traditional} compares the accuracy of our  estimators
and the (more complex) estimators in~\cite{kay_89,Fitz_94,Luise_Reggiannini_95}.
For small
frequency offset, the NM estimator has a slightly better accuracy than other estimators at moderate SNRs.
In general, our family of  SL estimators with different block lengths $v$
are very competitive across all SNRs considered here while maintaining
low computational complexity that allows for real-time application.

\begin{figure}[t]
    \centerline{\includegraphics[width=3.15in]{accuracy_NM_SL_traditional.png}}
    \caption{Accuracy of the SL, the NM and the traditional estimators ($L_0=32$, $M=2$, $M\delta=0.01$)}
    \label{fig:accuracy_NM_SL_traditional}
    \end{figure}

\begin{figure}[t]
  \centerline{\includegraphics[width=3.15in]{ROC_new.png}}
  \caption{Receiver operating characteristics (ROC) of the sequential detector ($M=2$, $M\delta=0.01$)}
  \label{fig:Receiver operating characteristics}
\end{figure}


\subsection{Simulation Results for Detection}

% Figure~\ref{fig:Generalized correlation} shows the performance of sequential detector
% in~\eqref{eq:generalized_corr} at each received signal windows. Note, because of pulse sha-ping,
% the autocorrelation property of the preamble sequence is decreased, which results
% the correlations at adjacent windows near the position of the preamble decay slow and thus make much challenges
% to choose threshold to make correct dicision.

% The solution is to adjust the detection algorithm by finding the local maximum of the correlation
% instead of just comparing the correlation with the threshold at each window. Note, when count for the ratio of false alarm and detection (for determining the threshold),
% the two positions should be counted as latter.


Fig.~\ref{fig:Receiver operating characteristics} shows the receiver
operating characteristics (ROC) of the detection algorithm. 
The better accuracy of SL estimation with partial integration also
improves the detection performance  at low SNR.
For example, at $\SI{-2}{\dB}$ SNR, $\gamma=0.44$, the false alarm probability $P_{FA}$ of SL with $v=32$ is reduced $0.2\%$ and 
the detection probability $P_{D}$ is $5\%$ larger compared with the SL with $v{=}1$. 
Fig.~\ref{fig:Receiver operating characteristics} also shows the detector does not work well at $\SI{-4}{\dB}$ SNR if only $32$ symbols of preamble are used;  
The performance is significantly improved by doubling the length of
the preamble. The two curves marked LRT at $\SI{-4}{\dB}$ SNR with different $v$
are obtained by plugging in the true frequency and phase offsets as an upper bound for comparing with 
the results of the GLRT.   

% \begin{figure}[t]
%   \centerline{\includegraphics[width=3.15in]{ROC_new.png}}
%   \caption{Receiver operating characteristics (ROC) of the sequential detector ($M=2$, $M\delta=0.01$)}
%   \label{fig:Receiver operating characteristics}
% \end{figure}



\section{Implementation on Software-defined Radio}%
\label{sec:implementation_on_SDR}

\begin{figure}[t]
    \centerline{\includegraphics[width=3.4in]{SDR_receiver_horizontal.png}}
    \caption{Block diagram for implementing the proposed algorithm in TBB}
    \label{fig:SDR_receiver}
    \end{figure}

% To make our proposed algorithm work at a very high sample rate,
% the first improvement of algorithm is by using pipeline.
% Threading Building Blocks (TBB) is a well-known C++
% library that enables parallel programming on multicore processor~\cite{Michael_19}.
To demonstrate the practicality of our algorithms, we have implemented them in C++ on a general
purpose processor (GPP).
The different aspects of the algorithm are mapped to logical nodes in a pipelined, parallel processing architecture
using Threading Building Blocks (TBB)~\cite{Michael_19}.

Fig.~\ref{fig:SDR_receiver} shows a simple block diagram
illustrating the pipeline for the complete joint detection and
estimation process.
Due to space constraints, we cannot give details for each node; the
computations performed by the nodes relate closely to the expressions
throughout this paper
% some of which are
as
indicated in Fig.~\ref{fig:SDR_receiver}.

It is necessary to discuss the node for computing the phasor estimate $\hat{\xi}$ (node V). 
The numerator of~\eqref{eq:opt_xi} performs a time-varying
convolution,
which cannot be computed efficiently via FFT.
Our solution to increasing the computational efficiency is to further
parallelize this block.
Specifically,~\eqref{eq:opt_xi} can be computed in three stages:
1.~Compute  segments of $\sum r_ns_n^*$ in  parallel;
2.~still in parallel, each segment is then phase-shifted by $\hat{\delta}_{SL}$
multiplied by the middle index of the 
segment.
3. Sum all segments together.
In other words, \eqref{eq:opt_xi} is computed more efficiently as
\begin{equation}
  \label{eq:refined_opt_S}
  ||\bm{s}||^2\cdot\hat{\xi} \approx \sum_{i=0}^{L-1} e^{-j\pi \hat{\delta}\frac{N(2i+1)}{L}}
  \sum_{n=iN/L}^{(i+1)N/L-1}r_ns_n^*,
\end{equation}
where $L$ is the number of segments processed in parallel.
As a result,
the efficiency of computing $\hat{\xi}$ is improved by $L^2$.

% We now show the results of implementing the proposed algorithm
% by using
Our goal is to assess the highest sample rate that a
highly parallel, pipelined implementation of our algorithms can sustain. 
The signals are transmitted and received between two universal software radio peripherals (USRP);
these are connected by \SI[per-mode=symbol]{5}{\giga\bits\per\second} Ethernet cables to
laptops.
On the receiver side, 
the CPU includes 6 cores and 12 threads with a \SI{4.5}{\giga\hertz}
clock frequency.

Each node in Fig.~\ref{fig:SDR_receiver} was benchmarked (using Google benchmark) to measure their throughput.
The results for each node are shown 
in Table~\ref{table:BM_function_nodes}.
The times displayed in the table
refer to processing a  buffer of size 512 samples.
The throughput of the 
pipeline is dominated by the node with the longest execution time.
Thus, the  throughput of our current implementation is bounded by
$\SI{512}{\sample}/\SI{52343}{\nano\second}  \approx \SI[per-mode=symbol]{9.78}{\mega\sample\per\second}$.
In operation, the throughput of the algorithm was measured to be in
the range of
\SI[per-mode=symbol]{9.5}{\mega\sample\per\second} to \SI[per-mode=symbol]{10.5}{\mega\sample\per\second} with latency near \SI{292}{\micro\second}.
The detection algorithm is very robust and the false alarm probability is near 0 at moderate SNR.

\begin{table}[t]
    \caption{Benchmark results of nodes in Fig.~\ref{fig:SDR_receiver} with buffer size 512}  % title of Table
    \centering % used for centering table
    \begin{tabular}{c c c c} % centered columns (4 columns)
    \hline\hline %inserts double horizontal lines
    Node name & Time (ns) & CPU (ns) & Iterations \\ [0.5ex] % inserts table
    %heading
    \hline % inserts single horizontal line
    I. Buffer  & 3885 & 3885 & 200135 \\ % inserting body of the table
    II. $r_m^*r_{m-k}$  & 3252 & 3252 & 216211 \\
    III. $r_m^*r_{m-k}s_{m-k}^*s_m$ & 52343 & 52342 & 13388 \\
    IV. $\hat{\delta}_{SL}^{(1)}$ & 8432 & 8432 & 86480 \\
    V. $\hat{\xi}$ & 52309 & 52309 & 13359 \\
    VI. $||\bm{r}||$ & 19638 & 19638 & 34872 \\ % [1ex] adds vertical space
    VII. $\rho(p)$ & 16217 & 16217 & 42283 \\
    VIII. detection & 3669 & 3669 & 199976 \\
    IX. carrier synchronization & 3075 & 3075 & 222154  \\ [1ex]
    \hline
    \end{tabular}
    \label{table:BM_function_nodes} % is used to refer this table in the text
  \end{table}



\section{Conclusion}

A family of algorithms for joint detection and carrier synchronization
suitable for LPI/LPD communications was presented.
These algorithms are designed for low complexity while achieving good
performance over a wide range of SNR.\@
We implemented our algorithms on a SDR platform and demonstrated that
they can sustain several \si{\mega\hertz} sample rate.

\bibliographystyle{ieeetr}
\bibliography{ref.bib}

\end{document}



%%% Local Variables:
%%% mode: latex
%%% TeX-master: t
%%% End:
