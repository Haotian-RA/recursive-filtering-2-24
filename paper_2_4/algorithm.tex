\section{The Proposed Algorithm}
\label{sec:algorithm}

The form of a general second order IIR filter is given by

\begin{equation}
    \label{eq:recursive_filter}
    % y[n] = x[n] + b_1x[n-1] + b_2x[n-2] + a_1y[n-1] + a_2y[n-2]
    y_n = x_n + b_1x_{n-1} + b_2x_{n-2} + a_1y_{n-1} + a_2y_{n-2}
\end{equation}
where $x_n$ and $y_n$ denote the input and output, respectively.
% missing some contents
The proposed algorithm in this paper processes multiple blocks (SIMD vectors) of data simultaneously
so that it utilizes both intra-block and inter-block parallelism. Define $M$ be the length of SIMD vector.
The input data of multiple blocks is exactly distributed in a squared matrix ($M$ by $M$), i.e.,

% a good example in this paper 
% \begin{equation}
% \left[\begin{array}{cccc|c}
% a_{11}&a_{12}&\cdots &a_{1n}&b_1\\
% a_{21}&a_{22}&\cdots &a_{2n}&b_2\\
% \vdots & &\ddots &\vdots \\
% a_{n1}&a_{n2}&\cdots &a_{nn}&b_n\\
% \end{array}\right]
% \end{equation}

% $$
% C=\left[
% \begin{array}{c ;{2pt/2pt} c}
% A & B \\ \hdashline[2pt/2pt]
% C & D
% \end{array}\right]
% $$

\begin{equation*}
    \label{eq:matrix_X}
    % \bm{X} = \left[\begin{array}{c|c|c|c}
    % x[0] & x[M] & \cdots & x[M^2{-}M] \\ 
    % x[1] & x[M{+}1] & \cdots & x[M^2{-}M{+}1] \\
    % \vdots & \vdots & \ddots & \vdots \\
    % x[M{-}1] & x[2M{-}1] &\cdots & x[M^2{-}1] \\
    % \end{array}\right]
    \bm{X} = \left[\begin{array}{c|c|c|c}
        x_0 & x_M & \cdots & x_{M^2-M} \\ 
        x_1 & x_{M+1} & \cdots & x_{M^2-M+1} \\
        \vdots & \vdots & \ddots & \vdots \\
        x_{M-1} & x_{2M-1} &\cdots & x_{M^2-1} \\
        \end{array}\right]
\end{equation*}
where each block of data is aligned with column and seperated by line.
We use $\bm{X}$ to denote the matrix version of input data. Furthermore, $\bm{X}[n]$
means the $n$th block and $\bm{X}[n][m]$ means the $m$th sample at $n$th block
of input matrix $\bm{X}$.
% missing some contents talking about dependency

To deal with the dependency problem of second order IIR filter equation, the algorithm
that computes the solution of finite impulse response (FIR), the particular and homogeneous solutions apart and 
add them to get the complete solutions is proposed.
% about homo and particular solutions and fir

\subsection{Finite Impulse Response (FIR)}

To compute \eqref{eq:recursive_filter}, we start with 

\begin{equation}
    \label{eq:fir_filter}
    v_n = x_n + b_1x_{n-1} + b_2x_{n-2} 
\end{equation}
where $v_n$ is the output of second order FIR filter. 

\subsubsection{Non-transposed multi-block filtering of FIR}

It is easy to apply the multi-block filtering method on FIR filter 
because of no dependency. The function that computes the output of second order FIR filter 
with non-transposed matrix of data is given by

\begin{equation}
    \label{eq:FIR_block_filtering_wo_trans}
    \begin{aligned}
        % \bm{V} &{=} \left[\begin{array}{c|c|c}
        %     V[0] & \cdots & V[M^2{-}M] \\ 
        %     V[1] & \cdots & V[M^2{-}M{+}1] \\
        %     \vdots & \ddots & \vdots \\
        %     V[M{-}1] &\cdots & V[M^2{-}1] \\
        %     \end{array}\right]
        % = \bm{X} + b_1\bm{X}_{-1} + b_2\bm{X}_{-2} \\
        % &{=} \bm{X} + b_1\left[\begin{array}{c|c|c}
        %     V_0 & \cdots & V_{M^2{-}M} \\ 
        %     V_1 & \cdots & V_{M^2{-}M{+}1} \\
        %     \vdots & \ddots & \vdots \\
        %     V_{M{-}1} &\cdots & V_{M^2{-}1} \\
        %     \end{array}\right] +
        %     b_1\left[\begin{array}{c|c|c}
        %         V_0 & \cdots & V_{M^2{-}M} \\ 
        %         V_1 & \cdots & V_{M^2{-}M{+}1} \\
        %         \vdots & \ddots & \vdots \\
        %         V_{M{-}1} &\cdots & V_{M^2{-}1} \\
        %         \end{array}\right]
        \bm{V} &= \left[\begin{array}{c|c|c|c}
            v_0 & v_M & \cdots & v_{M^2-M} \\ 
            v_1 & v_{M+1} & \cdots & v_{M^2-M+1} \\
            \vdots & \vdots & \ddots & \vdots \\
            v_{M-1} & v_{2M-1} &\cdots & v_{M^2-1} \\
            \end{array}\right] \\
            &= \bm{X} + b_1\left[\begin{array}{c|c|c|c}
                x_{-1} & x_{M-1} & \cdots & x_{M^2-M-1} \\ 
                x_0 & x_M & \cdots & x_{M^2-M} \\
                \vdots & \vdots & \ddots & \vdots \\
                x_{M-2} & x_{2M-2} &\cdots & x_{M^2-2} \\
                \end{array}\right]  \\
            &+ b_2\left[\begin{array}{c|c|c|c}
                x_{-2} & x_{M-2} & \cdots & x_{M^2-M-2} \\ 
                x_{-1} & x_{M-1} & \cdots & x_{M^2-M-1} \\
                \vdots & \vdots & \ddots & \vdots \\
                x_{M-3} & x_{2M-3} &\cdots & x_{M^2-3} \\
                \end{array}\right]  \\
            % &= \bm{X} + b_1\bm{X}_{-1} + b_2\bm{X}_{-2}
    \end{aligned}
\end{equation}
% where $\bm{X}_{-k}[n] = \left[ x_{nM-k} ~ x_{nM-k+1} ~ \cdots ~ x_{(n+1)M-k-1}\right]^T$ denotes
% the $n$th block of $\bm{X}$ shuffled with the previous $k$ input data. 
where $\bm{V}$ denotes the matrix version of output data of FIR.
\eqref{eq:FIR_block_filtering_wo_trans}
can be computed among multiple blocks exploiting both intra-block and inter-block parallelism. The execution
procedure of SIMD operation is as follows:

1. Load and broadcast two coefficients $b_1$ and $b_2$ in SIMD vector.

2. Load the first block (SIMD vector) of input $\bm{X}$ and shuffle it twice by $x_{-1}$, $x_{-2}$ to get
the initial conditions. 

3. Compute two fused multiply and add (FMA) operations to get the first output block $\bm{V}[0]$.

4. Repeat step 2 and 3 by shuffling with the last two input data of previous block for $M{-}1$ times 
to get the entire $\bm{V}$.

5. Store the last block of $\bm{X}$ in the shift register as the initial conditions for next FIR filtering.

\subsubsection{Transposed multi-block filtering of FIR}
Next, We provide another solution to computing the output of second order FIR filter, which is based on transposed
matrix version of data. The function also exhibits both intra-block and inter-block parallelism and is given by

\begin{equation}
    \label{eq:FIR_block_filtering_w_trans}
    \begin{aligned}
        % \bm{V} &{=} \left[\begin{array}{c|c|c}
        %     V[0] & \cdots & V[M^2{-}M] \\ 
        %     V[1] & \cdots & V[M^2{-}M{+}1] \\
        %     \vdots & \ddots & \vdots \\
        %     V[M{-}1] &\cdots & V[M^2{-}1] \\
        %     \end{array}\right]
        % = \bm{X} + b_1\bm{X}_{-1} + b_2\bm{X}_{-2} \\
        % &{=} \bm{X} + b_1\left[\begin{array}{c|c|c}
        %     V_0 & \cdots & V_{M^2{-}M} \\ 
        %     V_1 & \cdots & V_{M^2{-}M{+}1} \\
        %     \vdots & \ddots & \vdots \\
        %     V_{M{-}1} &\cdots & V_{M^2{-}1} \\
        %     \end{array}\right] +
        %     b_1\left[\begin{array}{c|c|c}
        %         V_0 & \cdots & V_{M^2{-}M} \\ 
        %         V_1 & \cdots & V_{M^2{-}M{+}1} \\
        %         \vdots & \ddots & \vdots \\
        %         V_{M{-}1} &\cdots & V_{M^2{-}1} \\
        %         \end{array}\right]
        \bm{V}^T &= \left[\begin{array}{c|c|c|c}
            v_0 & v_1 & \cdots & v_{M-1} \\ 
            v_M & v_{M+1} & \cdots & v_{2M-1} \\
            \vdots & \vdots & \ddots & \vdots \\
            v_{M^2-M} & v_{M^2-M+1} &\cdots & v_{M^2-1} \\
            \end{array}\right] \\
            &= \bm{X}^T + b_1\left[\begin{array}{c|c|c|c}
                x_{-1} & x_0 & \cdots & x_{M-2} \\ 
                x_{M-1} & x_M & \cdots & x_{2M-2} \\
                \vdots & \vdots & \ddots & \vdots \\
                x_{M^2-M-1} & x_{M^2-M} &\cdots & x_{M^2-2} \\
                \end{array}\right]  \\
            &+ b_2\left[\begin{array}{c|c|c|c}
                x_{-2} & x_{-1} & \cdots & x_{M-3} \\ 
                x_{M-2} & x_{M-1} & \cdots & x_{2M-3} \\
                \vdots & \vdots & \ddots & \vdots \\
                x_{M^2-M-2} & x_{M^2-M-1} &\cdots & x_{M^2-3} \\
                \end{array}\right]  \\
            % &= \bm{X}^T + b_1\bm{X}_{-1}^T + b_2\bm{X}_{-2}^T
    \end{aligned}
\end{equation}

Compared with non-transposed data matrix, the initial conditions for each block 
already exist at the previous two blocks except for the first block. Thus, to compute
\eqref{eq:FIR_block_filtering_w_trans}, only two shuffles are needed to provide the initial
conditions for the first block. The execution procedure is as follows:

1. Load and broadcast two coefficients $b_1$ and $b_2$ in SIMD vector.

2. Load the last two blocks of input $\bm{X}^T$ and shuffle those by $x_{-2}$, $x_{-1}$ respectively to get
the initial conditions for the first block.

3. Apply two FMA operations among the current block and the previous two blocks for $M$ times to get the entire $\bm{V}^T$.

4. Store (another shuffle) the last two input data located at the last two blocks in the shift register as the initial conditions for next FIR filtering.

\subsection{Infinite Impulse Response (IIR)}

After computing \eqref{eq:fir_filter}, the general equation of second order IIR filter in \eqref{eq:recursive_filter} is
simplified as

\begin{equation}
    \label{eq:iir_filter}
    y_n = v_n + a_1y_{n-1} + a_2y_{n-2} 
\end{equation}
We extend \eqref{eq:iir_filter} for the first couple of samples as follows:

\begin{equation}
    \label{eq:iir_filter_with_first_samples}
    \begin{aligned}
    & y_0 = \fbox{$v_0$} + a_1y_{-1} + a_2y_{-2} \\
    & y_1 = v_1 + a_1y_0 + a_2y_{-1} = \fbox{$v_1 + a_1v_0$} + (a_1^2+a_2)y_{-1}{+}a_1a_2y_{-2} \\
    & y_2 = v_2 + a_1y_1 + a_2y_0 = \fbox{$v_2 + a_1(v_1+a_1v_0) + a_2v_0$} + \\
    & \quad\quad\quad\quad\quad\quad\quad\quad\quad a_1^2(a_1y_{-1}+a_2y_{-2})+2a_1a_2y_{-1}+a_2^2y_{-2} \\
    & \quad\quad\quad\quad\quad\quad\quad\quad\quad\quad\quad\quad \vdots
    \end{aligned}
\end{equation}
where the terms in rectangle denote the particular solution and
the rest of terms are the homogeneous part of the second order IIR equation.
% missing some contents of literature
In this paper, we first calculate the particular solution by zeroing the homogenous part, where
the process is called zero initial condition (ZIC). After that, we correct the initial conditions (homogeneous part) for
getting the complete output. The latter process is called initial condition correction (ICC). Note, the above functions
are all processed by multi-block filtering method.

\subsubsection{Transposed zero initial condition (ZIC)}

From \eqref{eq:iir_filter_with_first_samples}, the function that computes the particular solution 
of second order IIR filter with transposed matrix of data is given by

\begin{equation}
    \label{eq:ZIC_w_trans}
    \begin{aligned}
        % \bm{V} &{=} \left[\begin{array}{c|c|c}
        %     V[0] & \cdots & V[M^2{-}M] \\ 
        %     V[1] & \cdots & V[M^2{-}M{+}1] \\
        %     \vdots & \ddots & \vdots \\
        %     V[M{-}1] &\cdots & V[M^2{-}1] \\
        %     \end{array}\right]
        % = \bm{X} + b_1\bm{X}_{-1} + b_2\bm{X}_{-2} \\
        % &{=} \bm{X} + b_1\left[\begin{array}{c|c|c}
        %     V_0 & \cdots & V_{M^2{-}M} \\ 
        %     V_1 & \cdots & V_{M^2{-}M{+}1} \\
        %     \vdots & \ddots & \vdots \\
        %     V_{M{-}1} &\cdots & V_{M^2{-}1} \\
        %     \end{array}\right] +
        %     b_1\left[\begin{array}{c|c|c}
        %         V_0 & \cdots & V_{M^2{-}M} \\ 
        %         V_1 & \cdots & V_{M^2{-}M{+}1} \\
        %         \vdots & \ddots & \vdots \\
        %         V_{M{-}1} &\cdots & V_{M^2{-}1} \\
        %         \end{array}\right]
        \bm{W}^T &= \left[\begin{array}{c|c|c|c}
            w_0 & w_1 & \cdots & w_{M-1} \\ 
            w_M & w_{M+1} & \cdots & w_{2M-1} \\
            \vdots & \vdots & \ddots & \vdots \\
            w_{M^2-M} & w_{M^2-M+1} &\cdots & w_{M^2-1} \\
            \end{array}\right] \\
            &= \bm{V}^T + a_1\left[\begin{array}{c|c|c|c}
                0 & v_0 & v_1+a_1v_0 & \cdots \\ 
                0 & v_M & v_{M+1}+a_1v_M & \cdots \\
                \vdots & \vdots & \vdots & \ddots \\
                0 & v_{M^2-M} & v_{M^2-M+1}+a_1v_{M^2-M} & \cdots \\
                \end{array}\right]  \\
            &+ a_2\left[\begin{array}{c|c|c|c}
                0 & 0 & v_0 & \cdots \\ 
                0 & 0 & v_M & \cdots \\
                \vdots & \vdots & \ddots & \ddots \\
                0 & 0 & v_{M^2-M} & \cdots \\
                \end{array}\right]  \\
            &= \bm{V}^T + a_1\left[\begin{array}{c|c|c|c|c}
                0 & w_0 & w_1 & \cdots & w_{M-2} \\ 
                0 & w_M & w_{M+1} & \cdots & w_{2M-2} \\
                \vdots & \vdots & \vdots & \ddots & \vdots \\
                0 & w_{M^2-M} & w_{M^2-M+1} & \cdots & w_{M^2-2} \\
                \end{array}\right]  \\
            &+ a_2\left[\begin{array}{c|c|c|c|c}
                0 & 0 & w_0 & \cdots & w_{M-3} \\ 
                0 & 0 & w_M & \cdots & w_{2M-3} \\
                \vdots & \vdots & \ddots & \ddots & \ddots \\
                0 & 0 & w_{M^2-M} & \cdots & w_{M^2-3} \\
                \end{array}\right]  \\
    \end{aligned}
\end{equation}
where $\bm{W}^T$ is the transposed output matrix of ZIC. Note, \eqref{eq:ZIC_w_trans} doesn't compute
the complete particular solution of \eqref{eq:iir_filter_with_first_samples}. For example, for each sample in the first block, the compensations (multiplications by $a_1$ and $a_2$) are two empty blocks.
This means the samples except for $w_0$ will lack of some information of past few samples due to the recursive property, e.g.,
the complete solution of $w_m$ should be a combination of $v_0$, $v_1$, $\cdots$, $v_m$. Thus, when we do
initial condition correction (ICC), each block should be corrected based on the last two output samples of the previous
block instead of the original states $y_{-1}$ and $y_{-2}$ only. We will explain more details later in section ICC.
The reason for doing \eqref{eq:ZIC_w_trans} is to leverage the intra-block parallelism.

Also note, because of the intrinsic dependency problem of IIR filter, inter-block parallelism can't be achieved simultaneously
when process \eqref{eq:ZIC_w_trans}. The execution procedure is as follows:

1. Load and broadcast two coefficients $a_1$ and $a_2$ in SIMD vector.

2. load the first block of $\bm{V}^T$ and store it as the first block in the output matrix $\bm{W}^T$.

3. load the second block of $\bm{V}^T$ and apply one FMA with the previous output block to get the second block of $\bm{W}^T$. 

4. load the rest blocks for $M{-}2$ times and apply two FMAs with the previous two output blocks to get the entire $\bm{W}^T$.

\subsubsection{Non-transposed zero initial condition}
If we re-write \eqref{eq:ZIC_w_trans} with non-transposed data version, i.e.,

\begin{equation}
    \label{eq:ZIC_wo_trans}
    \begin{aligned}
        % \bm{V} &{=} \left[\begin{array}{c|c|c}
        %     V[0] & \cdots & V[M^2{-}M] \\ 
        %     V[1] & \cdots & V[M^2{-}M{+}1] \\
        %     \vdots & \ddots & \vdots \\
        %     V[M{-}1] &\cdots & V[M^2{-}1] \\
        %     \end{array}\right]
        % = \bm{X} + b_1\bm{X}_{-1} + b_2\bm{X}_{-2} \\
        % &{=} \bm{X} + b_1\left[\begin{array}{c|c|c}
        %     V_0 & \cdots & V_{M^2{-}M} \\ 
        %     V_1 & \cdots & V_{M^2{-}M{+}1} \\
        %     \vdots & \ddots & \vdots \\
        %     V_{M{-}1} &\cdots & V_{M^2{-}1} \\
        %     \end{array}\right] +
        %     b_1\left[\begin{array}{c|c|c}
        %         V_0 & \cdots & V_{M^2{-}M} \\ 
        %         V_1 & \cdots & V_{M^2{-}M{+}1} \\
        %         \vdots & \ddots & \vdots \\
        %         V_{M{-}1} &\cdots & V_{M^2{-}1} \\
        %         \end{array}\right]
        &\bm{W} = \left[\begin{array}{c|c|c|c}
            w_0 & w_M & \cdots & w_{M^2-M} \\ 
            w_1 & w_{M+1} & \cdots & w_{M^2-M+1} \\
            \vdots & \vdots & \ddots & \vdots \\
            w_{M-1} & w_{2M-1} &\cdots & w_{M^2-1} \\
            \end{array}\right] = \bm{V} \\
            &+ a_1\left[\begin{array}{c|c|c|c}
                0 & 0 & \cdots & 0 \\ 
                v_0 & v_M & \cdots & v_{M^2-M} \\
                v_1{+}a_1v_0 & v_{M+1}{+}a_1v_M & \cdots & v_{M^2-M+1}{+}a_1v_{M^2-M} \\
                \vdots & \vdots & \ddots & \vdots \\
                \end{array}\right]  \\
            &+ a_2\left[\begin{array}{c|c|c|c}
                0 & 0 & \cdots & 0 \\ 
                0 & 0 & \cdots & 0 \\
                v_0 & v_M & \cdots & v_{M^2-M} \\
                \vdots & \vdots & \ddots & \vdots \\
                \end{array}\right]  \\
    \end{aligned}
\end{equation}
We can see that \eqref{eq:ZIC_wo_trans} can no longer be computed among multiple blocks. This is because
the dependency problem now occurs inside each block instead of happening among blocks as \eqref{eq:ZIC_w_trans}.
However, \eqref{eq:ZIC_wo_trans} can still achieve intra-block and inter-block parallelism by block-filtering method based on \eqref{eq:iir_filter_with_first_samples} as shown below

\begin{equation}
    \label{eq:vector_ZIC_wo_trans}
    \begin{aligned}
        % \bm{V} &{=} \left[\begin{array}{c|c|c}
        %     V[0] & \cdots & V[M^2{-}M] \\ 
        %     V[1] & \cdots & V[M^2{-}M{+}1] \\
        %     \vdots & \ddots & \vdots \\
        %     V[M{-}1] &\cdots & V[M^2{-}1] \\
        %     \end{array}\right]
        % = \bm{X} + b_1\bm{X}_{-1} + b_2\bm{X}_{-2} \\
        % &{=} \bm{X} + b_1\left[\begin{array}{c|c|c}
        %     V_0 & \cdots & V_{M^2{-}M} \\ 
        %     V_1 & \cdots & V_{M^2{-}M{+}1} \\
        %     \vdots & \ddots & \vdots \\
        %     V_{M{-}1} &\cdots & V_{M^2{-}1} \\
        %     \end{array}\right] +
        %     b_1\left[\begin{array}{c|c|c}
        %         V_0 & \cdots & V_{M^2{-}M} \\ 
        %         V_1 & \cdots & V_{M^2{-}M{+}1} \\
        %         \vdots & \ddots & \vdots \\
        %         V_{M{-}1} &\cdots & V_{M^2{-}1} \\
        %         \end{array}\right]
        \bm{W} &= \left[\begin{array}{c|c|c|c}
            w_0 & w_M & \cdots & w_{M^2-M} \\ 
            w_1 & w_{M+1} & \cdots & w_{M^2-M+1} \\
            \vdots & \vdots & \ddots & \vdots \\
            w_{M-1} & w_{2M-1} &\cdots & w_{M^2-1} \\
            \end{array}\right]  \\
            &= \left[\begin{array}{ccccc}
                1 & 0 & \cdots & \cdots & 0 \\ 
                a_1 & 1 & \ddots & 0  & \vdots \\
                a_1^2+a_2 & a_1 & 1 & \ddots & \vdots \\
                a_1^3+2a_1a_2 & a_1^2+a_2 & a_1 & 1 & 0 \\
                \vdots & \vdots & \vdots & \vdots & \ddots\vdots \\
                \end{array}\right] \bm{V} = \bm{H}\bm{V} \\  
    \end{aligned}
\end{equation}
where \eqref{eq:vector_ZIC_wo_trans} performs a matrix multiplication and $\bm{H}$ denotes
the transition matrix from $\bm{V}$ to $\bm{W}$. $\bm{H}$ can be pre-computed
by an impulse response, $\left[0 ~ 1~ \dashline ~ a_1 ~ a_1^2+a_2 ~ (a_1^2+a_2)a_1{+}a_1a_2~\cdots\right]$,
where each response is obtained by multiplying the previous two responses with $a_2$ and $a_1$.
The vertical dashed line is used to seperate the initial state and the resulting response sequence,
which is denoted by $\bm{h}_1 = \left[a_1 ~ a_1^2+a_2 ~ (a_1^2+a_2)a_1{+}a_1a_2~\cdots\right]$.
The length of $\bm{h}_1$ is $M$.
The execution procedure of \eqref{eq:vector_ZIC_wo_trans} is as follows:

1. Pre-calculate and load the transition matrix $\bm{H}$. Note, this won't take the run time.

2. Load and broadcast every sample in the first block of $\bm{V}$. Apply $M$ times FMAs with each SIMD vectors at the same index in $\bm{H}$
and accumulate them together to get the first block of $\bm{W}$.

3. Repeat step 2 for $M{-}1$ times for the rest blocks to get the entire $\bm{W}$.

Note, the resulting $\bm{W}$ from \eqref{eq:vector_ZIC_wo_trans} is exactly the same as 
$\bm{W}^T$ from \eqref{eq:ZIC_w_trans} without matrix transpose.

\subsubsection{Non-transposed initial condition correction (ICC)}

After computing the particular solution of second order IIR filter, we now compensate for the 
initial conditions to get the complete solution. Recall that the output matrix of ZIC in either \eqref{eq:ZIC_w_trans} or
\eqref{eq:vector_ZIC_wo_trans} only has partial particular solution. Thus, based on \eqref{eq:iir_filter_with_first_samples} and along with ZIC,
the function that computes the homogeneous solution of second order
IIR filter with non-transposed matrix of data is given by 

\begin{equation}
    \label{eq:ICC_wo_trans}
    \begin{aligned}
        % \bm{V} &{=} \left[\begin{array}{c|c|c}
        %     V[0] & \cdots & V[M^2{-}M] \\ 
        %     V[1] & \cdots & V[M^2{-}M{+}1] \\
        %     \vdots & \ddots & \vdots \\
        %     V[M{-}1] &\cdots & V[M^2{-}1] \\
        %     \end{array}\right]
        % = \bm{X} + b_1\bm{X}_{-1} + b_2\bm{X}_{-2} \\
        % &{=} \bm{X} + b_1\left[\begin{array}{c|c|c}
        %     V_0 & \cdots & V_{M^2{-}M} \\ 
        %     V_1 & \cdots & V_{M^2{-}M{+}1} \\
        %     \vdots & \ddots & \vdots \\
        %     V_{M{-}1} &\cdots & V_{M^2{-}1} \\
        %     \end{array}\right] +
        %     b_1\left[\begin{array}{c|c|c}
        %         V_0 & \cdots & V_{M^2{-}M} \\ 
        %         V_1 & \cdots & V_{M^2{-}M{+}1} \\
        %         \vdots & \ddots & \vdots \\
        %         V_{M{-}1} &\cdots & V_{M^2{-}1} \\
        %         \end{array}\right]
        &\bm{Y} = \left[\begin{array}{c|c|c|c}
            y_0 & y_M & \cdots & y_{M^2-M} \\ 
            y_1 & y_{M+1} & \cdots & y_{M^2-M+1} \\
            \vdots & \vdots & \ddots & \vdots \\
            y_{M-1} & y_{2M-1} &\cdots & y_{M^2-1} \\
            \end{array}\right] = \bm{W} + \\
            &\left[\begin{array}{cc}
                a_1 & a_2 \\ 
                a_1^2+a_2 & a_1a_2 \\
                a_1^3+2a_1a_2 & a_1^2a_2+a_2^2 \\
                \vdots & \vdots
                \end{array}\right]  
                \left[\begin{array}{cccc}
                    y_{-1} & y_{M-1} & \cdots & y_{M^2-M-1} \\ 
                    y_{-2} & y_{M-2} & \cdots & y_{M^2-M-2} \\
                    \end{array}\right] \\
            &= \bm{W} + \bm{h}\bm{Y}_p
    \end{aligned}
\end{equation}
where $\bm{h}$ denotes the transition matrix for the pre-state matrix $\bm{Y}_p$, and it is composed by
two impulse response sequences, $\bm{h}_1$ and $\bm{h}_2$, where $\bm{h}_2$ is the resulting response sequence of 
$\left[1 ~ 0~ \dashline ~ a_2 ~ a_1a_2 ~ a_1^2a_2{+}a_2^2~\cdots\right]$,
following the same rule as $\bm{h}_1$. Compared with \eqref{eq:iir_filter_with_first_samples}, because the
output samples of ZIC from \eqref{eq:ZIC_w_trans} or \eqref{eq:vector_ZIC_wo_trans} only has partial particular solution,
we can't compensate for every block by the original initial conditions $y_{-1}$ and $y_{-2}$.
Instead, we correct each block based on the pre-states of the start sample at each block, e.g., $\bm{W}[1]$ should
be compensated by $\bm{Y}[0][M-1]$ and $\bm{Y}[0][M-2]$. Furthermore,
\eqref{eq:ICC_wo_trans} exploits intra-block parallelism and multi-block filtering method. 
The execution procedure is as follows:

1. Pre-calculate and load the transition matrix $\bm{h}$.

2. Load and broadcast the first two initial conditions $y_{-1}$, $y_{-2}$ and apply two FMAs with two SIMD vectors 
in $\bm{h}$ to get the first block of $\bm{Y}$.

3. Repeat step 2 for $M{-}1$ times with two initial conditions, which are the last two samples of the prvious output block,
to get the entire $\bm{Y}$.

4. Store the last block of $\bm{Y}$ in the shift register as the initial
conditions for next IIR filtering.

\subsubsection{Transposed initial condition correction (ICC)}

To compute the homogeneous solution with transposed matrix of data, we re-write \eqref{eq:ICC_wo_trans}
in transposed version, i.e.,

\begin{equation}
    \label{eq:ICC_w_trans}
    \begin{aligned}
        % \bm{V} &{=} \left[\begin{array}{c|c|c}
        %     V[0] & \cdots & V[M^2{-}M] \\ 
        %     V[1] & \cdots & V[M^2{-}M{+}1] \\
        %     \vdots & \ddots & \vdots \\
        %     V[M{-}1] &\cdots & V[M^2{-}1] \\
        %     \end{array}\right]
        % = \bm{X} + b_1\bm{X}_{-1} + b_2\bm{X}_{-2} \\
        % &{=} \bm{X} + b_1\left[\begin{array}{c|c|c}
        %     V_0 & \cdots & V_{M^2{-}M} \\ 
        %     V_1 & \cdots & V_{M^2{-}M{+}1} \\
        %     \vdots & \ddots & \vdots \\
        %     V_{M{-}1} &\cdots & V_{M^2{-}1} \\
        %     \end{array}\right] +
        %     b_1\left[\begin{array}{c|c|c}
        %         V_0 & \cdots & V_{M^2{-}M} \\ 
        %         V_1 & \cdots & V_{M^2{-}M{+}1} \\
        %         \vdots & \ddots & \vdots \\
        %         V_{M{-}1} &\cdots & V_{M^2{-}1} \\
        %         \end{array}\right]
        &\bm{Y}^T = \left[\begin{array}{c|c|c|c}
            y_0 & y_1 & \cdots & y_{M-1} \\ 
            y_M & y_{M+1} & \cdots & y_{2M-1} \\
            \vdots & \vdots & \ddots & \vdots \\
            y_{M^2-M} & y_{M^2-M+1} &\cdots & y_{M^2-1} \\
            \end{array}\right] = \bm{W}^T + \\
            &\left[\begin{array}{c|c}
                y_{-1} & y_{-2} \\ 
                y_{M-1} & y_{M-2} \\
                \vdots & \vdots \\
                y_{M^2-M-1} & y_{M^2-M-2}
                \end{array}\right]  
                \left[\begin{array}{cccc}
                    a_1 & a_1^2+a_2 & a_1^3+2a_1a_2 & \cdots \\ 
                    a_2 & a_1a_2 & a_1^2a_2+a_2^2 & \cdots \\
                    \end{array}\right] \\
            &= \bm{W}^T + \bm{Y}_p^T\bm{h}^T
    \end{aligned}
\end{equation}
Compared with \eqref{eq:ICC_wo_trans}, we need one extra step to process \eqref{eq:ICC_w_trans} 
that is to compute the two initial-condition vectors in $\bm{Y}_p^T$. The two initial conditions
can be computed at first since they only depend on $\bm{W}$ and the two pre-states $y_{-2}$ and $y_{-1}$.
Thus, the function of calculating $\bm{Y}_p^T$ is given by

\begin{equation}
    \label{eq:ICC_init_w_trans}
    \begin{aligned}
        % \bm{V} &{=} \left[\begin{array}{c|c|c}
        %     V[0] & \cdots & V[M^2{-}M] \\ 
        %     V[1] & \cdots & V[M^2{-}M{+}1] \\
        %     \vdots & \ddots & \vdots \\
        %     V[M{-}1] &\cdots & V[M^2{-}1] \\
        %     \end{array}\right]
        % = \bm{X} + b_1\bm{X}_{-1} + b_2\bm{X}_{-2} \\
        % &{=} \bm{X} + b_1\left[\begin{array}{c|c|c}
        %     V_0 & \cdots & V_{M^2{-}M} \\ 
        %     V_1 & \cdots & V_{M^2{-}M{+}1} \\
        %     \vdots & \ddots & \vdots \\
        %     V_{M{-}1} &\cdots & V_{M^2{-}1} \\
        %     \end{array}\right] +
        %     b_1\left[\begin{array}{c|c|c}
        %         V_0 & \cdots & V_{M^2{-}M} \\ 
        %         V_1 & \cdots & V_{M^2{-}M{+}1} \\
        %         \vdots & \ddots & \vdots \\
        %         V_{M{-}1} &\cdots & V_{M^2{-}1} \\
        %         \end{array}\right]
        &\left[\begin{array}{c}
            y_{M-2}  \\ 
            y_{2M-2}  \\
            \vdots \\
            y_{M^2-2} \\ \hline 
            y_{M-1}  \\ 
            y_{2M-1}  \\
            \vdots \\
            y_{M^2-1} \\
            \end{array}\right] = 
            \left[\begin{array}{cc}
                \bm{h}_{2,M-2} & \bm{h}_{1,M-2} \\ 
                \bm{h}_{2,M-2}\bm{h}_{2,M-2} & \bm{h}_{1,M-2}\bm{h}_{2,M-2} \\
                +\bm{h}_{2,M-1}\bm{h}_{1,M-2} & +\bm{h}_{1,M-1}\bm{h}_{1,M-2} \\
                \vdots & \vdots \\ \hline
                \bm{h}_{2,M-1} & \bm{h}_{1,M-1} \\ 
                \bm{h}_{2,M-2}\bm{h}_{2,M-1} & \bm{h}_{1,M-2}\bm{h}_{2,M-1} \\
                +\bm{h}_{2,M-1}\bm{h}_{1,M-1} & +\bm{h}_{1,M-1}\bm{h}_{1,M-1} \\
                \vdots & \vdots
                \end{array}\right]
                \left[\begin{array}{c}
                    y_{-2}  \\ 
                    y_{-1}  \\
                    \end{array}\right] + \\
            &\left[\begin{array}{ccc|ccc}
                1 & 0 & \cdots & 0 & \cdots & \cdots \\ 
                \bm{h}_{2,M-2} & 1 & 0 & \bm{h}_{1,M-2} & 0 & 0 \\
                % \bm{h}_2[M{-}2]\bm{h}_2[M{-}2]+\bm{h}_2[M{-}1]\bm{h}_1[M{-}2]
                \bm{h}_{2,M-2}\bm{h}_{2,M-2}+ & \ddots & \ddots\vdots & \bm{h}_{1,M-2}\bm{h}_{2,M-2}+ & \ddots & \ddots\vdots \\
                \bm{h}_{2,M-1}\bm{h}_{1,M-2}  & & & \bm{h}_{1,M-1}\bm{h}_{1,M-2} \\
                \vdots & \vdots & \ddots\vdots & \vdots & \vdots & \ddots\vdots \\ \hline
                0 & \cdots & \cdots & 1 & 0 & \cdots \\ 
                \bm{h}_{2,M-1} & 0 & 0 & \bm{h}_{1,M-1} & 1 & 0 \\
                % \bm{h}_2[M{-}2]\bm{h}_2[M{-}2]+\bm{h}_2[M{-}1]\bm{h}_1[M{-}2]
                \bm{h}_{2,M-2}\bm{h}_{2,M-1}+ & \ddots & \ddots\vdots & \bm{h}_{1,M-2}\bm{h}_{2,M-1}+ & \ddots & \ddots\vdots \\
                \bm{h}_{2,M-1}\bm{h}_{1,M-1}  & & & \bm{h}_{1,M-1}\bm{h}_{1,M-1} \\
                \vdots & \vdots & \ddots\vdots & \vdots & \vdots & \ddots\vdots \\      
            \end{array}\right] \\ 
            &\cdot \left[\begin{array}{cccc|cccc}
                w_{M-2} & w_{2M-2} & \cdots & w_{M^2-2} & w_{M-1} &  & \cdots & w_{M^2-1} \\ 
                \end{array}\right]^T \\
            &= \bm{D}\bm{y}_p + \bm{T} \cdot 
            \left[\begin{array}{c}
                \bm{W}^T[M{-}2]  \\ 
                \bm{W}^T[M{-}1]  \\
                \end{array}\right]
            % &\left[\begin{array}{c|c}
            %     a_1 & a_2 \\ 
            %     a_1^2+a_2 & a_1a_2 \\
            %     a_1^3+2a_1a_2 & a_1^2a_2+a_2^2 \\
            %     \vdots & \vdots
            %     \end{array}\right]  
            %     \left[\begin{array}{cccc}
            %         y_{-1} & y_{M-1} & \cdots & y_{M^2-M-1} \\ 
            %         y_{-2} & y_{M-2} & \cdots & y_{M^2-M-2} \\
            %         \end{array}\right] \\
            % &= \bm{W} + \bm{h}\bm{Y}_p
    \end{aligned}
\end{equation}
where $\bm{h}_{x,n}$ is equivalent to $\bm{h}_x[n]$ and 
$\bm{D}$ and $\bm{T}$ are two transition matrices for the pre-state vector $\bm{y}_p$
and the corresponding part of $\bm{W}$. 
$\bm{D}$ and $\bm{T}$ can be pre-computed by the following two joint
impulse responses, which are 

1. $\Big[1 ~ 0~ \dashline ~\bm{h}_{2,M-2} ~\bm{h}_{2,M-1} ~\bm{h}_{2,M-2}\bm{h}_{2,M-2}{+}\bm{h}_{2,M-1}\bm{h}_{1,M-2} $

$\quad\quad\quad\quad\quad\quad\quad\quad\quad\quad ~ \bm{h}_{2,M-2}\bm{h}_{2,M-1}{+}\bm{h}_{2,M-1}\bm{h}_{1,M-1} \cdots\Big]$

2. $\Big[0 ~ 1~ \dashline ~\bm{h}_{1,M-2} ~\bm{h}_{1,M-1} ~\bm{h}_{1,M-2}\bm{h}_{2,M-2}{+}\bm{h}_{1,M-1}\bm{h}_{1,M-2} $

$\quad\quad\quad\quad\quad\quad\quad\quad\quad\quad ~ \bm{h}_{1,M-2}\bm{h}_{2,M-1}{+}\bm{h}_{1,M-1}\bm{h}_{1,M-1} \cdots\Big]$
The first position of initial state corresponds to $\bm{h}_2$ while the second one points to $\bm{h}_1$. Furthermore,
each of the above two impulse sequences calculates a pair of impulse responses simultaneously and interleavedly.
The current pair of impulse response are calculated based on the previous pair and the former determines multiplying $\bm{h}_x$ of position $M{-}2$  
while the latter points to $M{-}1$. 

Note, after computing \eqref{eq:ICC_init_w_trans}, the pre-state matrix $\bm{Y}_p^T$ in \eqref{eq:ICC_w_trans}
can be then obtained by shuffling with $y_{-1}$ and $y_{-2}$. Furthermore,
since $\bm{h}^T$ can be pre-computed,
the process of computing \eqref{eq:ICC_w_trans} achieves both intra-block and inter-block parallelism.
The execution procedure of \eqref{eq:ICC_init_w_trans} and \eqref{eq:ICC_w_trans} is as follows:

1. Pre-compute and load two transition matrices $\bm{D}$ and $\bm{T}$.

2. Load and broadcast $y_{-1}$, $y_{-2}$ and each sample in the last two blocks of $\bm{W}$, then apply
$4M{+}4$ FMAs to get the last two blocks of $\bm{Y}^T$ following \eqref{eq:ICC_init_w_trans}.

3. Shuffle the obtained two output blocks at step 2 with $y_{-1}$ and $y_{-2}$ respectively to get the pre-state matrix $\bm{Y}_p^T$.

4. Load and broadcast each element in $\bm{h}^T$ and apply two FMAs for $M{-}2$ times to get the entire $\bm{Y}^T$ following \eqref{eq:ICC_w_trans}.

\subsection{Matrix transpose}


Since the proposed algorithm accepts both transposed and non-transposed data at each processing phase, it is necessary to discuss
how matrix transpose is performed in SIMD operation. In this paper, the matrix transpose for multiple blocks of data is executed as follows:

1. Swap lower left part and upper right of the matrix, i.e.,

\begin{equation*}
    \begin{aligned}
    \label{eq:matrix_trans_step_1}
    % \bm{X} = \left[\begin{array}{c|c|c|c}
    % x[0] & x[M] & \cdots & x[M^2{-}M] \\ 
    % x[1] & x[M{+}1] & \cdots & x[M^2{-}M{+}1] \\
    % \vdots & \vdots & \ddots & \vdots \\
    % x[M{-}1] & x[2M{-}1] &\cdots & x[M^2{-}1] \\
    % \end{array}\right]
    &\bm{X} \rightarrow \\
    &\left[\begin{array}{c|c|c|c|c}
        x_0 & x_M & & x_{M/2} &  \\ 
        x_1 & x_{M+1} & & x_{M/2+1} & \\
        \vdots & \vdots & & \vdots & \\
        x_{M/2-1} & x_{M+M/2-1} & \cdot\cdot & x_{M-1} & \cdot\cdot \\
        x_{M/2\cdot M} & x_{(M/2+1)\cdot M} & & x_{M/2\cdot M+M/2} & \\
        x_{M/2\cdot M+1} & x_{(M/2+1)\cdot M+1} & & x_{M/2\cdot M+M/2+1} & \\
        \vdots & \vdots & & \vdots & \\
        x_{M/2\cdot (M+1)-1} & x_{(M/2+1)\cdot M+M/2-1} & & x_{(M/2+1)M-1} & \\
        \end{array}\right]
    \end{aligned}
\end{equation*}

2. Swap lower left part and upper right within each quadrant,

\begin{equation*}
    \label{eq:matrix_trans_step_2}
    % \bm{X} = \left[\begin{array}{c|c|c|c}
    % x[0] & x[M] & \cdots & x[M^2{-}M] \\ 
    % x[1] & x[M{+}1] & \cdots & x[M^2{-}M{+}1] \\
    % \vdots & \vdots & \ddots & \vdots \\
    % x[M{-}1] & x[2M{-}1] &\cdots & x[M^2{-}1] \\
    % \end{array}\right]
    \rightarrow 
    \left[\begin{array}{c|c|c|c}
        x_0 & x_M & x_2 &  \\ 
        \vdots & \vdots & \vdots & \\
        x_{M/4-1} & x_{M+M/4-1} & x_{M/2-1} & \\
        x_{M/4\cdot M} & x_{(M/4+1)\cdot M} & x_{M/4\cdot M+2} &  \\
        \vdots & \vdots & \vdots & \\
        x_{M/4\cdot M+M/4-1} & x_{M/4\cdot M+5M/4-1} & x_{M/4\cdot M+M/2-1} & \cdot\cdot \\
        x_{M/2\cdot M} & x_{(M/2+1)\cdot M} & x_{M/2\cdot M+2} &  \\ 
        \vdots & \vdots & \vdots & \\
        x_{M/2\cdot M+M/4-1} & x_{M/2\cdot M+5M/4-1} & x_{M/2\cdot M+M/2-1} & \\
        x_{3M/2\cdot M} & x_{(3M/2+1)\cdot M} & x_{3M/2\cdot M+2} &  \\
        \vdots & \vdots & \vdots & \\
        x_{3M/2\cdot M+M/4-1} & x_{3M/2\cdot M+5M/4-1} & x_{3M/2\cdot M+M/2-1} & \\    
    \end{array}\right]
\end{equation*}

3. Repeat step 2 with each sub-quadrant until step 4 is ready

4. Swap anti-diagonally adjacent elements to get $\bm{X}^T$.

It can be seen that the entire process of matrix transpose in SIMD operation requires $M{\log_2}M$
shuffles.
